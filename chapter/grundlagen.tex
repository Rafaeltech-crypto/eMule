\chapter{Grundlagen}
\label{cha:Grundlagen}

Das folgende Kapitel fasst die für die vorliegende Arbeit benötigten theoretischen Grundlagen zusammen. Hierzu wird in Kapitel 2.1 die Geschichte der Elektrofahrzeuge näher betrachtet. Anschließend wird in Kapitel 2.2 auf Normen zur Zeichnung von Schaltzeichen eingegangen. Abschließend wird in Kapitel 2.3 das zur Erstellung der Dokumentation genutzte Tool Autodesk Fusion 360 durchleuchtet.
\section{Geschichte der Elektrofahrzeuge}

Die Geschichte der Elektrofahrzeuge ist ein faszinierendes Kapitel in der Entwicklung der Mobilität. Obwohl Elektrofahrzeuge heute als Zukunftstechnologie gelten, reichen ihre Ursprünge weit zurück und sind eng mit den Anfängen des Automobilbaus verknüpft.

\subsection*{Die Anfänge im 19. Jahrhundert}

Bereits in der ersten Hälfte des 19. Jahrhunderts wurden die Grundlagen für Elektrofahrzeuge geschaffen. Der Schotte Robert Anderson baute in den 1830er Jahren eines der ersten elektrisch betriebenen Fahrzeuge. Es handelte sich um ein einfaches Fahrzeug mit einer nicht wiederaufladbaren Batterie. In den folgenden Jahrzehnten wurden Elektrofahrzeuge durch die Entwicklung von wiederaufladbaren Batterien und Elektromotoren immer praktikabler.\autocite{vattenfall_elektroauto_geschichte} 
Einen wichtigen Beitrag leistete der Franzose Gaston Planté, der 1859 den ersten funktionsfähigen Bleiakkumulator entwickelte. Diese wiederaufladbare Batterie ermöglichte den kontinuierlichen Betrieb von Elektromotoren und legte den Grundstein für die spätere Entwicklung von Elektrofahrzeugen. \autocite{cosmos_gaston_plante}

\subsection*{Die Blütezeit der Elektrofahrzeuge um 1900}

Um die Jahrhundertwende erfreuten sich Elektrofahrzeuge großer Beliebtheit. Sie waren leiser, sauberer und einfacher zu bedienen als die damals üblichen Fahrzeuge mit Dampf- oder Verbrennungsmotoren. Vor allem in Städten wurden Elektroautos aufgrund ihrer geringen Reichweite und einfachen Handhabung bevorzugt eingesetzt. \autocite{energyprofi_elektromobile_geschichte} Marken wie Baker Electric und Detroit Electric prägten diese Ära. \autocite{einfacheauto_elektroauto_geschichte}

Elektrofahrzeuge hatten zu dieser Zeit bedeutende Marktvorteile. Während Verbrennungsmotoren oft manuell gekurbelt werden mussten und unangenehm laut waren, konnten Elektrofahrzeuge mit einem einfachen Schalter gestartet werden. \autocite{enbw_elektroautos_vorteile_nachteile} Die Reichweiten von etwa 50 bis 100 Kilometern pro Batterieladung reichten für den städtischen Einsatz vollkommen aus. \autocite{adac_stromverbrauch_elektroautos}

\subsection*{Der Rückgang durch den Verbrennungsmotor}

Die Dominanz der Elektrofahrzeuge begann jedoch im ersten Drittel des 20. Jahrhunderts zu schwinden. Wesentliche Faktoren dafür waren:
\begin{itemize}
	\item Die Erfindung des elektrischen Anlassers durch Charles Kettering im Jahr 1912, der die Handkurbel bei Verbrennungsmotoren überflüssig machte. \autocite{greelane_charles_kettering}
	\item Die zunehmende Verfügbarkeit von billigem Erdöl, das Kraftstoffe für Verbrennungsmotoren erschwinglich machte. \autocite{tanke_guenstig_oelpreise}
	\item Die Massenproduktion von Fahrzeugen mit Verbrennungsmotor durch Henry Ford, die die Kosten für Autos drastisch senkte. \autocite{ardalpha_henry_ford_geschichte}
	\item Bis in die 1930er Jahre waren Elektrofahrzeuge weitgehend vom Markt verdrängt. \autocite{ardalpha_henry_ford_geschichte}
\end{itemize}

\subsection*{Wiederbelebung im 20. Jahrhundert}

Die Energiekrisen der 1970er Jahre und das wachsende Umweltbewusstsein führten zu einem erneuten Interesse an Elektrofahrzeugen. \autocite{daswissen_oelkrise_1970er} Automobilhersteller experimentierten mit Prototypen, um Alternativen zu fossilen Brennstoffen zu entwickeln. \autocite{energieleben_ev1_geschichte} In dieser Phase entstanden Fahrzeuge wie der General Motors EV1, der 1996 eingeführt wurde. \autocite{insideevs_gm_ev1}
Trotz seiner technischen Fortschritte wurde die Produktion jedoch nach wenigen Jahren eingestellt.\autocite{energieleben_ev1_geschichte} 

\subsection*{Die Renaissance der Elektrofahrzeuge im 21. Jahrhundert}
Der Beginn des 21. Jahrhunderts markierte eine neue Ära für Elektrofahrzeuge. Fortschritte in der Batterietechnologie, insbesondere die Entwicklung von Lithium-Ionen-Akkus, machten Elektroautos leistungsfähiger und alltagstauglicher. \autocite{sonepar_batterien_2023} Gleichzeitig führten Umweltauflagen und staatliche Förderprogramme zu einer verstärkten Nachfrage.\newline
Ein entscheidender Wendepunkt war die Gründung von Tesla Motors im Jahr 2003. Mit dem Tesla Roadster, der 2008 auf den Markt kam, bewies das Unternehmen, dass Elektrofahrzeuge nicht nur umweltfreundlich, sondern auch leistungsstark und attraktiv sein können. Dies ebnete den Weg für weitere Modelle wie den Nissan Leaf, den BMW i3 und die elektrische Version des Volkswagen Golf. \autocite{insidetesla_tesla_geschichte}

\subsection*{Herausforderungen und Perspektiven}
Trotz der Erfolge stehen Elektrofahrzeuge weiterhin vor Herausforderungen.  Die Infrastruktur für Ladestationen muss ausgebaut werden, um eine flächendeckende Versorgung zu gewährleisten. \autocite{statista_ladeinfrastruktur_elektroautos} Zudem sind die Produktionskosten von Batterien nach wie vor hoch, obwohl sie durch Skaleneffekte und technologische Fortschritte stetig sinken. \autocite{automobilproduktion_produktionskosten_elektroautos} \newline
Die Perspektiven für Elektrofahrzeuge sind dennoch vielversprechend. Die fortschreitende Entwicklung von Feststoffbatterien und die Integration erneuerbarer Energien in die Stromerzeugung könnten die Elektromobilität nachhaltig vorantreiben. Politische Initiativen wie das Verbot von Verbrennungsmotoren in einigen Ländern ab 2035 unterstreichen den globalen Wandel hin zu emissionsfreien Fahrzeugen. \autocite{fraunhofer_batterie_rohstoffe}

\section{Lithium-Ionen-Batterien}
Lithium-Ionen-Batterien werden aufgrund ihrer kompakten Bauweise bereits seit Jahren in der Computertechnik eingesetzt. Ihr Anwendungsbereich erstreckt sich dabei von Smartphones bis hin zu Laptops. Angesichts des bevorstehenden Verbots von Blei in Fahrzeugen gewinnt ihr Einsatz auch im Automobilsektor zunehmend an Bedeutung und wird perspektivisch unverzichtbar.\newline
Eine Lithium-Ionen-Batterie mit einer Nennspannung von X Volt besteht aus in Reihe geschalteten Zellen, wodurch sich die Zellspannungen addieren, während die Gesamtkapazität durch die Kapazität der schwächsten Zelle begrenzt bleibt. Für größere Kapazitätsanforderungen werden Zellen parallel geschaltet, wodurch sich die Kapazitäten der einzelnen Zellen addieren, während die Spannung unverändert bleibt. Die Bewertung einer solchen Batterie erfolgt klassischerweise anhand ihrer nominalen Kapazität, der gespeicherten elektrischen Energie und ihrer Leistung.

\begin{figure}
	\centering
	\includegraphics[width=0.7\linewidth]{images/Li-Zelle}
	\caption{Aufbau einer Lithium-Ionen-Zelle}
	\label{fig:li-zelle}
\end{figure}

Eine einzelne Lithium-Ionen-Zelle \ref{fig:li-zelle} besteht grundlegend aus einer Anode, einer Kathode, einem Separator, Ableitern und einem Elektrolyten. Der positive Bereich der Zelle befindet sich auf der Seite der Anode, die aus einem Ableiter besteht, der mit einer Schicht aus Graphit, also Kohlenstoff, beschichtet ist. Als Material für den Ableiter wird üblicherweise Kupfer, seltener auch Nickel, verwendet. Die Kathode hingegen bildet das negative Element der Zelle und besteht aus einem Aluminium-Ableiter, der mit Materialien wie Lithium-Cobalt-Oxid, Lithium-Mangan-Oxid oder Lithium-Eisen-Phosphat beschichtet ist. Der Zwischenraum zwischen den beiden Elektroden ist mit einem flüssigen Elektrolyten gefüllt, der den Ionentransport zwischen den Elektroden ermöglicht. Dabei wird eine möglichst hohe Leitfähigkeit sichergestellt, um den Betrieb der Zelle in einem Temperaturbereich von -40 °C bis +80 °C zu gewährleisten. \autocite{Lith-Akku}
Ein Elektrolyt ist im Wesentlichen eine Flüssigkeit, die mit Leitsalzen angereichert ist, um den Ionentransport zu ermöglichen. Darüber hinaus muss das Elektrolyt eine hohe Stabilität aufweisen, um mehreren tausend Lade- und Entladezyklen standzuhalten. \autocite[S.61f.]{Korthauer.2013}
Der Separator bildet eine Trennschicht innerhalb des Elektrolyten zwischen den Elektroden einer Lithium-Ionen-Zelle. Er besteht in der Regel aus einer Membran oder einem Vliesstoff aus Materialien wie Glasfaser oder Kunststoffen und weist eine Porosität von etwa 40 \% auf. Seine besondere Eigenschaft ist die selektive Durchlässigkeit für Ionen, die für die Umwandlung von chemischer in elektrische Energie unerlässlich sind. Elektronen hingegen werden durch den Separator blockiert, um sicherzustellen, dass sie über externe Leitungen zu den Verbrauchern, wie beispielsweise einem Steuergerät, transportiert werden können. Nach ihrer Nutzung kehren die Elektronen über den externen Stromkreis in die Zelle zurück, wo sie auf die gegenüberliegende Seite zu den Ionen gelangen. \newline \autocite[S. 80]{Korthauer.2013}
Der Separator spielt eine entscheidende Rolle bei der Vermeidung interner Kurzschlüsse, die ohne ihn auftreten würden. Zusätzlich fördert er den Gasaustausch, indem er das Elektrolyt aufsaugt. Die physikalischen Eigenschaften des Separators, wie seine Dicke und Porosität, beeinflussen maßgeblich den Innenwiderstand der Zelle und tragen somit zur Gesamtleistung bei. \autocite[S. 80]{Korthauer.2013}
Beim Laden einer Lithium-Ionen-Batterie fungiert die positive Elektrode als Anode, während sie beim Entladen als Kathode dient. Der Ladevorgang erfolgt üblicherweise nach dem sogenannten CC-CV-Verfahren (Constant Current - Constant Voltage). Dabei wird zunächst ein konstanter Strom (Constant Current, CC) angelegt, bis die Batterie eine festgelegte Spannung erreicht. Anschließend wird die Spannung konstant gehalten (Constant Voltage, CV), wobei der Stromfluss progressiv abnimmt. Die Beendigung des Ladevorgangs erfolgt in der Regel durch eine vorgegebene Zeitbegrenzung oder das Erreichen einer definierten Stromschwelle. \autocite[S. 15]{Korthauer.2013}
Lithium-Ionen-Akkumulatoren sind stark temperaturabhängig, da bei sehr niedrigen Temperaturen der Innenwiderstand deutlich ansteigt. Dies ist auf die verlangsamten chemischen Reaktionen innerhalb der Zelle zurückzuführen. Darüber hinaus ist es essenziell, eine Überladung der Batterie zu vermeiden, da dies zu sogenannten Zerfallsreaktionen führen kann. Die Intensität dieser Reaktionen variiert je nach den verwendeten Materialien der Zellkomponenten und kann die Lebensdauer sowie die Sicherheit der Batterie erheblich beeinträchtigen. \autocite[S. 15f.]{Korthauer.2013}


\section{Normen zur Zeichnung von Schaltzeichen}

\subsection*{Entstehung und Bedeutung von Normen}
Normen haben ihren Ursprung in der industriellen Revolution, als der Bedarf an standardisierten Verfahren und Produkten exponentiell anstieg. Unterschiedliche Maße, Zeichnungen oder Bezeichnungen führten zu Missverständnissen, Ineffizienzen und Fehlern in der Fertigung und Kommunikation. Um diesem Chaos entgegenzuwirken, wurden Normen geschaffen, die als verbindliche Regelwerke dienen.

Normen ermöglichen eine einheitliche Sprache zwischen Ingenieuren, Herstellern und Anwendern. Sie sichern die Kompatibilität von Bauteilen, verbessern die Qualität und fördern den internationalen Handel. Im Kontext technischer Zeichnungen – insbesondere von Schaltzeichen – gewährleisten Normen, dass technische Pläne weltweit eindeutig verstanden werden können, unabhängig von Sprache oder regionalen Besonderheiten.

\subsection*{Die bekanntesten Normen für Schaltzeichen}

Drei der bekanntesten und am häufigsten verwendeten Normen für Schaltzeichen sind:
\begin{itemize}
\item \textbf{DIN-Normen (Deutschland): }Diese Normen, herausgegeben vom Deutschen Institut für Normung, sind insbesondere im deutschsprachigen Raum verbreitet. Sie umfassen eine breite Palette von Standards, darunter auch solche für elektrische, hydraulische und pneumatische Schaltzeichen.
\item \textbf{IEC-Normen (International):} Die International Electrotechnical Commission (IEC) ist für die Entwicklung global gültiger Standards verantwortlich. Die IEC 60617-Serie beispielsweise definiert Symbole für elektrotechnische Anlagen und Komponenten.
\item \textbf{ANSI-Normen (USA): }Das American National Standards Institute (ANSI) ist die dominierende Normierungsorganisation in den USA. ANSI-Zeichnungen sind häufig in nordamerikanischen Projekten anzutreffen.
\end{itemize}
Die Wahl der Norm hängt von der Region und dem Anwendungsfall ab. Während europäische Projekte häufig auf DIN- oder IEC-Normen basieren, dominieren ANSI-Normen in den USA.

\subsection*{Die DIN-Norm für Schaltzeichen im Detail}

Die DIN-Normen sind in Deutschland der zentrale Standard für die Erstellung technischer Zeichnungen und Schaltpläne. Besonders relevant ist die Norm DIN EN 60617, die elektrische Schaltzeichen beschreibt. Diese Norm wurde in Zusammenarbeit mit der IEC entwickelt, was die internationale Anschlussfähigkeit erleichtert.

Die DIN EN 60617 regelt detailliert:
\begin{itemize}
	\item \textbf{Die Darstellung von Bauelementen:} Elektronische Bauteile wie Widerstände, Kondensatoren oder Schalter haben klar definierte Symbole.
	\item \textbf{Das Layout von Schaltplänen:} Vorgaben für Linienführung, Anschlussstellen und Abstände zwischen Symbolen sorgen für Übersichtlichkeit.
	\item \textbf{Verbindungsleitungen:} Die Darstellung von Leitungen und Kreuzungen vermeidet Missverständnisse, beispielsweise durch eindeutige Markierungen bei Verbindungen.
\end{itemize}
Ein zentrales Ziel der DIN-Norm ist es, Komplexität zu reduzieren und eine intuitive Lesbarkeit zu fördern. Zusätzlich berücksichtigt die Norm auch neuere Technologien und Entwicklungen, wodurch sie immer wieder aktualisiert wird.

Durch die Einhaltung der DIN-Norm können Ingenieure sicherstellen, dass ihre Schaltpläne sowohl in der eigenen Organisation als auch international korrekt interpretiert werden. Normen sind daher nicht nur ein Werkzeug der Standardisierung, sondern auch ein Mittel zur Qualitätssteigerung und zur Vereinfachung technischer Prozesse.

\section{Autodesk Fusion 360}
\label{Autodesk}
Autodesk Fusion 360 ist eine integrierte Plattform für computergestütztes Design (CAD), Fertigung (CAM) und technische Analyse (CAE), die als Cloud-basierte Lösung entwickelt wurde. Sie erlaubt es, mechanische und elektronische Designprozesse zu vereinen, und bietet damit Ingenieuren, Designern und Entwicklern eine zentrale Plattform für die Produktentwicklung. Im Folgenden wird zunächst die Unternehmensgeschichte von Autodesk als Entwickler dieser Software beleuchtet, bevor die Kernfunktionen und speziellen Funktionen zur Erstellung elektronischer Schaltpläne detailliert werden.
\subsection{Installationsaleitung}
Anleitung zur Erstellung eines Studentenaccounts und zum Herunterladen von Fusion 360 Electronics.

\subsection*{Erstellung eines Autodesk-Studentenaccounts}

Zur Nutzung von Fusion 360 Electronics ist die Erstellung eines Autodesk-Studentenaccounts erforderlich. Dies ermöglicht den kostenlosen Zugriff auf die Software.

\paragraph{ Registrierung}
\begin{itemize}
	\item Zugriff auf die Registrierungsseite: \href{https://accounts.autodesk.com/register?resume=/as/fMRyxxIM12/resume/as/authorization.ping&ack=uWlmiJuqQqVaAQjGdojc8Qxit4KVdorZ}{\underline{Autodesk Registrierungsseite}}.
	\item Ausfüllen des Formulars mit den notwendigen Informationen:
	\begin{itemize}
		\item Vor- und Nachname
		\item Gültige E-Mail-Adresse
		\item Passwort entsprechend den Sicherheitsrichtlinien
	\end{itemize}
\end{itemize}

\paragraph{ Bestätigung der E-Mail-Adresse}
\begin{itemize}
	\item Nach dem Absenden des Formulars wird eine E-Mail zur Bestätigung empfangen.
	\item Öffnen der E-Mail und Klicken auf den Bestätigungslink zur Verifizierung der Adresse.
\end{itemize}

\paragraph{ Vervollständigung der Profilinformationen}
\begin{itemize}
	\item Anmeldung im Autodesk-Konto.
	\item Angabe weiterer Informationen wie Institution, Studienrichtung und Studienjahr zur Bestätigung des Studentenstatus.
\end{itemize}

\paragraph*{ Verifizierung des Studentenstatus}
\begin{itemize}
	\item Hochladen eines Dokuments, das die Immatrikulation belegt (z. B. eine Studienbescheinigung).
	\item Autodesk prüft die Dokumente innerhalb weniger Tage und sendet eine Bestätigung per E-Mail.
\end{itemize}

\subsection*{Herunterladen und Installieren von Fusion 360 Electronics}

\paragraph{Zugriff auf den Download-Bereich}
\begin{itemize}
	\item Nach erfolgreicher Verifizierung des Accounts erfolgt die Anmeldung und Navigation zur \href{https://www.autodesk.com/education/home}{\underline{Autodesk Education Community}}.
	\item Auswahl von Fusion 360 aus der Liste der verfügbaren Software.
\end{itemize}

\paragraph{ Download und Installation}
\subsection*{Windows}
\begin{itemize}
	\item Beachten Sie bei der Auswahl der Downloaddatei die Unterschiede zwischen den Softwareversionen für die verschiedenen Windows-Betriebssysteme. Diese unterscheiden sich in der Versionsnummer (z. B. \glqq Windows 11\grqq {}) und in den Bit-Versionen (32- und 64-Bit).
	\item Schritte zur Identifikation der Windows-Version:
		\begin{itemize}
		\item[1.] Drücke die Tastenkombination Windows-Taste + I, um die Einstellungen zu öffnen.
		\item[2.] Gehe zu System → Info.
		\item[3.] Unter Windows-Spezifikationen findest du die genaue Version und Edition von Windows (z. B. \glqq Windows 11 Pro\grqq {}, \glqq Version 22H2\grqq {}).
		\end{itemize}
	\item Schritte zur Identifikation der Bit-Version:
		\begin{itemize}
		\item[1.] Drücke die Tastenkombination Windows-Taste + I, um die Einstellungen zu öffnen.
		\item[2.] Gehe zu System → Info.
		\item[3.] 	Unter Gerätespezifikationen → Systemtyp steht z. B. \glqq 64-Bit-Betriebssystem\grqq {}.
		\end{itemize}
	\item Klicken auf „Jetzt herunterladen“ und Befolgen der Anweisungen auf dem Bildschirm.
	\item Nach Abschluss des Downloads Öffnen der Installationsdatei und Befolgen der Installationsanweisungen.
\end{itemize}
\subsection*{macOS}
\begin{itemize}
	\item Beachten Sie bei der Auswahl der Downloaddatei die Unterschiede zwischen der Softwareversion für Betriebssysteme mit Apple Silicon Prozessor und Intel Prozessor.
	\item Schritte zur Identifikation des verbauten Prozessors:
	\begin{itemize}
		\item[1.] Klicke oben links auf das Apple-Symbol.
		\item[2.] Wähle "Über diesen Mac".
		\item[3.] Schaue im Fenster, das sich öffnet: \newline
		Wenn dort \glqq Chip\grqq {} steht, gefolgt von z. B. \glqq Apple M1\grqq {} oder \glqq Apple M2\grqq {}, ist ein Apple Silicon Prozessor verbaut.\newline
		Wenn dort \glqq Prozessor\grqq {} steht, gefolgt von einem Intel-Prozessor (z. B. \glqq Intel Core i5\grqq {}), ist ein Intel-Prozessor in dem Mac verbaut.
	\end{itemize}
	\item Klicken auf „Jetzt herunterladen“ und Befolgen der Anweisungen auf dem Bildschirm.
	\item Nach Abschluss des Downloads Öffnen der Installationsdatei und Befolgen der Installationsanweisungen.
\end{itemize}

\paragraph*{Aktivierung der Education-Lizenz}
\begin{itemize}
	\item Beim ersten Start von Fusion 360 erfolgt die Eingabe der Anmeldeinformationen.
	\item Die Software erkennt automatisch den Studentenstatus und aktiviert die entsprechende Lizenz.
\end{itemize}





\subsection{Historie und Entwicklung}
Autodesk, Inc. wurde 1982 von John Walker und einer Gruppe von Programmierern gegründet und spezialisierte sich schnell auf Softwarelösungen für Architektur, Ingenieurwesen und digitale Medien. \autocite{wikipedia_autodesk}
Die Veröffentlichung von AutoCAD im Jahr 1982 setzte einen wichtigen Meilenstein für die computergestützte Konstruktion und wurde zur führenden CAD-Software für Architekten und Ingenieure weltweit.\autocite{wikipedia_autocad_version_history}


Mit dem Aufkommen neuer Anforderungen in der Fertigungsindustrie und der Integration von Elektronik in mechanische Systeme begann Autodesk, eine neue Art von Software zu entwickeln. Ziel war es, die Mechanik- und Elektronikentwicklung auf einer Plattform zu vereinen und kollaboratives, Cloud-basiertes Arbeiten zu ermöglichen. Dies führte zur Einführung von Fusion 360 im Jahr 2013. \autocite{wikipedia_autodesk_deutsch}
Durch die Integration traditioneller CAD/CAM/CAE-Funktionen und die cloudbasierte Zusammenarbeit wurde Fusion 360 zu einem beliebten Werkzeug in der Produktentwicklung und verhalf Autodesk zu einer neuen Marktposition im Bereich der digitalen Fertigung.

