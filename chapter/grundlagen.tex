\chapter{Grundlagen}
\label{cha:Grundlagen}
\section{Autodesk Fusion 360}
\label{Autodesk}
Autodesk Fusion 360 ist eine integrierte Plattform für computergestütztes Design (CAD), Fertigung (CAM) und technische Analyse (CAE), die als Cloud-basierte Lösung entwickelt wurde. Sie erlaubt es, mechanische und elektronische Designprozesse zu vereinen, und bietet damit Ingenieuren, Designern und Entwicklern eine zentrale Plattform für die Produktentwicklung. Im Folgenden wird zunächst die Unternehmensgeschichte von Autodesk als Entwickler dieser Software beleuchtet, bevor die Kernfunktionen und speziellen Funktionen zur Erstellung elektronischer Schaltpläne detailliert werden.
\subsection{Installationsaleitung}
\subsection*{Windows}
\subsection*{Mac}
\subsection{Historie und Entwicklung}
Autodesk, Inc. wurde 1982 von John Walker und einer Gruppe von Programmierern gegründet und spezialisierte sich schnell auf Softwarelösungen für Architektur, Ingenieurwesen und digitale Medien. 
%@misc{wikipedia_autodesk,
%	author       = "{Wikipedia contributors}",
%	title        = "{Autodesk --- Wikipedia{,} The Free Encyclopedia}",
%	year         = 2024,
%	url          = "https://en.wikipedia.org/wiki/Autodesk",
%	note         = "[Online; accessed 11-November-2024]"}
Die Veröffentlichung von AutoCAD im Jahr 1982 setzte einen wichtigen Meilenstein für die computergestützte Konstruktion und wurde zur führenden CAD-Software für Architekten und Ingenieure weltweit.
%@misc{wikipedia_autocad_version_history,
%	author       = "{Wikipedia contributors}",
%	title        = "{AutoCAD version history --- Wikipedia{,} The Free Encyclopedia}",
%	year         = 2024,
%	url          = "https://en.wikipedia.org/wiki/AutoCAD_version_history",
%	note         = "[Online; accessed 11-November-2024]"}


Mit dem Aufkommen neuer Anforderungen in der Fertigungsindustrie und der Integration von Elektronik in mechanische Systeme begann Autodesk, eine neue Art von Software zu entwickeln. Ziel war es, die Mechanik- und Elektronikentwicklung auf einer Plattform zu vereinen und kollaboratives, Cloud-basiertes Arbeiten zu ermöglichen. Dies führte zur Einführung von Fusion 360 im Jahr 2013. 
%@misc{wikipedia_autodesk,
%	author       = {Wikipedia contributors},
%	title        = {Autodesk -- Wikipedia{,} Die freie Enzyklopädie},
%	year         = 2024,
%	url          = {https://de.wikipedia.org/wiki/Autodesk},
%	note         = {[Online; abgerufen am 13. November 2024]}}
Durch die Integration traditioneller CAD/CAM/CAE-Funktionen und die cloudbasierte Zusammenarbeit wurde Fusion 360 zu einem beliebten Werkzeug in der Produktentwicklung und verhalf Autodesk zu einer neuen Marktposition im Bereich der digitalen Fertigung.
\subsection{Grundfunktionen}
\subsection{Spezielle Funktionen zur Erstellung von Schaltplänen, Bestückungsplänen und Stromlaufplänen}

	\begin{itemize}
		\item Normen -> Din norm
		\item Autodesk
		\item wenn man strecken muss: kawasaki mule+definition elektrofzg
		\end{itemize}
Zielgerichtete theoretische Grundlagen, sowohl fachliche, wie auch methodische.

Zu den Grundlagen gehören z.~B. auch Details zur Problemstellung, der Stand der Technik und weitere Grundlagen, welche zur Konzeptausarbeitung, Umsetzung und Verifikation erforderlich sind.

Grundlagen haben immer einen Bezug zu den nachfolgenden Kapiteln. Diesen Bezug sollte man gelegentlich explizit herstellen, damit bereits in diesem Kapitel klar ist, wo und für was die Grundlagen gebraucht und angewandt werden.