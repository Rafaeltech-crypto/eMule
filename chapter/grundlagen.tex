\chapter{Foundations}
\label{cha:Grundlagen}

%Das folgende Kapitel fasst die für die vorliegende Arbeit benötigten theoretischen Grundlagen zusammen. Hierzu wird in Kapitel 2.1 die Geschichte der Elektrofahrzeuge näher betrachtet. Anschließend werden in Kapitel 2.2 auf Lithium-Ionen-Batterien, in Kapitel 2.3 auf Normen zur Zeichnung von Schaltzeichen eingegangen und abschließend in Kapitel 2.4 das zur Erstellung der Dokumentation genutzte Tool Autodesk Fusion 360 durchleuchtet.


The subsequent chapter provides a synopsis of the theoretical foundations that are prerequisite for the present thesis. The second chapter provides a more detailed examination of the history of electric vehicles. In the subsequent chapter, Chapter 2.2, the focus is on lithium-ion batteries. Chapter 2.3 then deals with standards for drawing circuit diagrams, and Chapter 2.4 analyses the Autodesk Fusion 360 tool used to create the documentation.
\section{The history of electric vehicles}

%Die Geschichte der Elektrofahrzeuge ist ein faszinierendes Kapitel in der Entwicklung der Mobilität. Obwohl Elektrofahrzeuge heute als Zukunftstechnologie gelten, reichen ihre Ursprünge weit zurück und sind eng mit den Anfängen des Automobilbaus verknüpft.
The history of electric vehicles is an intriguing chapter in the development of mobility. Despite the contemporary perception of electric vehicles as a technology that is poised for imminent widespread adoption, their origins can be traced back to the early days of automotive engineering, thus underscoring their deep-rooted history.
\subsection*{The beginnings in the 19th century}

%Bereits in der ersten Hälfte des 19. Jahrhunderts wurden die Grundlagen für Elektrofahrzeuge geschaffen. Der Schotte Robert Anderson baute in den 1830er Jahren eines der ersten elektrisch betriebenen Fahrzeuge. Es handelte sich um ein einfaches Fahrzeug mit einer nicht wiederaufladbaren Batterie. In den folgenden Jahrzehnten wurden Elektrofahrzeuge durch die Entwicklung von wiederaufladbaren Batterien und Elektromotoren immer praktikabler.
The foundations for electric vehicles were established in the early 19th century. The Scotsman Robert Anderson is widely regarded as one of the first to construct an electrically powered vehicle, a feat which was achieved in the 1830s. The vehicle was uncomplicated in design and was equipped with a non-rechargeable battery. In the subsequent decades, the practicality of electric vehicles increased significantly due to the advancement of rechargeable batteries and electric motors.\autocite{vattenfall_elektroauto_geschichte} 
%Einen wichtigen Beitrag leistete der Franzose Gaston Planté, der 1859 den ersten funktionsfähigen Bleiakkumulator entwickelte. Diese wiederaufladbare Batterie ermöglichte den kontinuierlichen Betrieb von Elektromotoren und legte den Grundstein für die spätere Entwicklung von Elektrofahrzeugen. 
An important contribution was made by the Frenchman Gaston Planté, who developed the first functional lead-acid accumulator in 1859. This rechargeable battery was pivotal in enabling the continuous operation of electric motors and laid the foundation for the subsequent development of electric vehicles.\autocite{cosmos_gaston_plante}

\subsection*{The heyday of electric vehicles around 1900}

%Um die Jahrhundertwende erfreuten sich Elektrofahrzeuge großer Beliebtheit. Sie waren leiser, sauberer und einfacher zu bedienen als die damals üblichen Fahrzeuge mit Dampf- oder Verbrennungsmotoren. Vor allem in Städten wurden Elektroautos aufgrund ihrer geringen Reichweite und einfachen Handhabung bevorzugt eingesetzt. 
Electric vehicles experienced a period of significant popularity around the turn of the century. In comparison with the prevalent steam or combustion engine vehicles of that era, the new vehicles were characterised by a quieter operation, a more pristine condition and a simplified operational process. In urban areas, electric cars were particularly favoured due to their limited range and ease of use.\autocite{energyprofi_elektromobile_geschichte} Brands like Baker Electric and Detroit Electric shaped this era \autocite{einfacheauto_elektroauto_geschichte}.

%Elektrofahrzeuge hatten zu dieser Zeit bedeutende Marktvorteile. Während Verbrennungsmotoren oft manuell gekurbelt werden mussten und unangenehm laut waren, konnten Elektrofahrzeuge mit einem einfachen Schalter gestartet werden. 
Electric vehicles had significant market advantages at the time. In contrast to the often laborious and sonically disagreeable process of manually engaging combustion engines, electric vehicles were initiated with a mere flick of a switch, thus facilitating a more expeditious and less strenuous operation.\autocite{enbw_elektroautos_vorteile_nachteile} %Die Reichweiten von etwa 50 bis 100 Kilometern pro Batterieladung reichten für den städtischen Einsatz vollkommen aus.
The ranges of around 50 to 100 kilometres per battery charge were perfectly adequate for urban use \autocite{adac_stromverbrauch_elektroautos}.
\clearpage

\subsection*{The decline due to the combustion engine}

%Die Dominanz der Elektrofahrzeuge begann jedoch im ersten Drittel des 20. Jahrhunderts zu schwinden. Wesentliche Faktoren dafür waren:
However, the dominance of electric vehicles began to wane in the first third of the 20th century. The main factors behind this were:
\begin{itemize}
	\item %Die Erfindung des elektrischen Anlassers durch Charles Kettering im Jahr 1912, der die Handkurbel bei Verbrennungsmotoren überflüssig machte. 
	The invention of the electric starter motor by Charles Kettering in 1912, which made the hand crank on combustion engines superfluous \autocite{greelane_charles_kettering}
	\item %Die zunehmende Verfügbarkeit von billigem Erdöl, das Kraftstoffe für Verbrennungsmotoren erschwinglich machte.
	The increasing availability of cheap crude oil, which made fuels for internal combustion engines affordable \autocite{tanke_guenstig_oelpreise}
	\item %Die Massenproduktion von Fahrzeugen mit Verbrennungsmotor durch Henry Ford, die die Kosten für Autos drastisch senkte. 
	The mass production of vehicles with internal combustion engines by Henry Ford, which drastically reduced the cost of cars \autocite{ardalpha_henry_ford_geschichte}
	%\item Bis in die 1930er Jahre waren Elektrofahrzeuge weitgehend vom Markt verdrängt. 
\end{itemize}
By the 1930s, electric vehicles had largely disappeared from the market	\autocite{ardalpha_henry_ford_geschichte}.
\subsection*{Revival in the 20th century}

%Die Energiekrisen der 1970er Jahre und das wachsende Umweltbewusstsein führten zu einem erneuten Interesse an Elektrofahrzeugen. 
The energy crises of the 1970s and growing environmental awareness led to a renewed interest in electric vehicles \autocite{daswissen_oelkrise_1970er}. %Automobilhersteller experimentierten mit Prototypen, um Alternativen zu fossilen Brennstoffen zu entwickeln. 
Car manufacturers experimented with prototypes to develop alternatives to fossil fuels \autocite{energieleben_ev1_geschichte}. %In dieser Phase entstanden Fahrzeuge wie der General Motors EV1, der 1996 eingeführt wurde. 
During this phase, vehicles such as the General Motors EV1, which was launched in 1996, were created \autocite{insideevs_gm_ev1}.
%Trotz seiner technischen Fortschritte wurde die Produktion jedoch nach wenigen Jahren eingestellt.
Despite its technical advances, however, production was discontinued after a few years \autocite{energieleben_ev1_geschichte}.

\subsection*{The renaissance of electric vehicles in the 21st century}
%Der Beginn des 21. Jahrhunderts markierte eine neue Ära für Elektrofahrzeuge. Fortschritte in der Batterietechnologie, insbesondere die Entwicklung von Lithium-Ionen-Akkus, machten Elektroautos leistungsfähiger und alltagstauglicher. 
The beginning of the 21st century marked a new era for electric vehicles. Advancements in battery technology, notably the development of lithium-ion batteries, have rendered electric cars more potent and more appropriate for daily utilisation.\autocite{sonepar_batterien_2023} 
%Gleichzeitig führten Umweltauflagen und staatliche Förderprogramme zu einer verstärkten Nachfrage.\newline
%Ein entscheidender Wendepunkt war die Gründung von Tesla Motors im Jahr 2003. Mit dem Tesla Roadster, der 2008 auf den Markt kam, bewies das Unternehmen, dass Elektrofahrzeuge nicht nur umweltfreundlich, sondern auch leistungsstark und attraktiv sein können. Dies ebnete den Weg für weitere Modelle wie den Nissan Leaf, den BMW i3 und die elektrische Version des Volkswagen Golf. 
Simultaneously, an increase in demand was precipitated by environmental regulations and government subsidy programmes.\newline
A significant turning point was marked by the establishment of Tesla Motors in 2003. The Tesla Roadster, which was launched on the market in 2008, demonstrated that electric vehicles could be both environmentally sustainable and aesthetically pleasing. This development subsequently influenced the introduction of other models, including the Nissan Leaf, the BMW i3, and the electric version of the Volkswagen Golf.\autocite{insidetesla_tesla_geschichte}

\subsection*{Challenges and perspectives}
%Trotz der Erfolge stehen Elektrofahrzeuge weiterhin vor Herausforderungen.  Die Infrastruktur für Ladestationen muss ausgebaut werden, um eine flächendeckende Versorgung zu gewährleisten.
Despite the successes, electric vehicles still face challenges.  The infrastructure for charging stations must be expanded to ensure nationwide coverage.\autocite{statista_ladeinfrastruktur_elektroautos} 
%Zudem sind die Produktionskosten von Batterien nach wie vor hoch, obwohl sie durch Skaleneffekte und technologische Fortschritte stetig sinken. 
Furthermore, the manufacturing expenses of batteries remain substantial, though they are gradually declining due to economies of scale and technological advancements.\autocite{automobilproduktion_produktionskosten_elektroautos} \newline
%Die Perspektiven für Elektrofahrzeuge sind dennoch vielversprechend. Die fortschreitende Entwicklung von Feststoffbatterien und die Integration erneuerbarer Energien in die Stromerzeugung könnten die Elektromobilität nachhaltig vorantreiben. Politische Initiativen wie das Verbot von Verbrennungsmotoren in einigen Ländern ab 2035 unterstreichen den globalen Wandel hin zu emissionsfreien Fahrzeugen.
Nonetheless, the outlook for electric vehicles appears to be favourable. The ongoing development of solid-state batteries and the integration of renewable energies into power generation could drive electric mobility forward in the long term. The global transition towards zero-emission vehicles has been highlighted by political initiatives, such as the prohibition on combustion engines in certain countries, effective from 2035.\autocite{fraunhofer_batterie_rohstoffe}

\section{Lithium-ion batteries}
%Lithium-Ionen-Batterien werden aufgrund ihrer kompakten Bauweise bereits seit Jahren in der Computertechnik eingesetzt. Ihr Anwendungsbereich erstreckt sich dabei von Smartphones bis hin zu Laptops. Angesichts des bevorstehenden Verbots von Blei in Fahrzeugen gewinnt ihr Einsatz auch im Automobilsektor zunehmend an Bedeutung und wird perspektivisch unverzichtbar.\newline
%Eine Lithium-Ionen-Batterie mit einer Nennspannung von X Volt besteht aus in Reihe geschalteten Zellen, wodurch sich die Zellspannungen addieren, während die Gesamtkapazität durch die Kapazität der schwächsten Zelle begrenzt bleibt. Für größere Kapazitätsanforderungen werden Zellen parallel geschaltet, wodurch sich die Kapazitäten der einzelnen Zellen addieren, während die Spannung unverändert bleibt. Die Bewertung einer solchen Batterie erfolgt klassischerweise anhand ihrer nominalen Kapazität, der gespeicherten elektrischen Energie und ihrer Leistung.

Lithium-ion batteries have been utilised within the domain of computer technology for an extended period, attributable to their compact configuration. The applications for these devices range widely, encompassing smartphones and laptops. In consideration of the imminent prohibition on the use of lead in vehicles, their utilisation is becoming progressively significant within the automotive sector. It is anticipated that their use will be indispensable in the future.\newline
A lithium-ion battery with a nominal voltage of X volts consists of cells connected in series. In this configuration, the cell voltages are added together, but the total capacity is limited by the capacity of the weakest cell. In scenarios where increased capacity is required, the cells are connected in parallel, resulting in the summation of their individual capacities without altering the voltage. The evaluation of such a battery is classically based on its nominal capacity, the stored electrical energy and its performance.

\begin{figure}[h]
	\centering
	\includegraphics[width=0.7\linewidth]{images/Li-Zelle}
	\caption{Structure of a lithium-ion cell \autocite{scinexx_akku_schattenseiten}}
	\label{fig:li-zelle}
\end{figure}

%Eine einzelne Lithium-Ionen-Zelle (siehe Abbildung \ref{fig:li-zelle}) besteht grundlegend aus einer Anode, einer Kathode, einem Separator, Ableitern und einem Elektrolyten. Der positive Bereich der Zelle befindet sich auf der Seite der Anode, die aus einem Ableiter besteht, der mit einer Schicht aus Graphit, also Kohlenstoff, beschichtet ist. Als Material für den Ableiter wird üblicherweise Kupfer, seltener auch Nickel, verwendet. Die Kathode hingegen bildet das negative Element der Zelle und besteht aus einem Aluminium-Ableiter, der mit Materialien wie Lithium-Cobalt-Oxid, Lithium-Mangan-Oxid oder Lithium-Eisen-Phosphat beschichtet ist. Der Zwischenraum zwischen den beiden Elektroden ist mit einem flüssigen Elektrolyten gefüllt, der den Ionentransport zwischen den Elektroden ermöglicht. Dabei wird eine möglichst hohe Leitfähigkeit sichergestellt, um den Betrieb der Zelle in einem Temperaturbereich von -40 °C bis +80 °C zu gewährleisten. 
As illustrated in \ref{fig:li-zelle}, a lithium-ion cell comprises an anode, a cathode, a separator, arresters and an electrolyte. The positive area of the cell is located on the side of the anode, which is composed of a trap coated with a layer of graphite, i.e. carbon. The material most frequently utilised for the arrester is copper; nickel is employed on less frequent occasions. Conversely, the cathode constitutes the negative element of the cell and consists of an aluminium arrester coated with materials such as lithium cobalt oxide, lithium manganese oxide or lithium iron phosphate. The space between the two electrodes is filled with a liquid electrolyte that facilitates the movement of ions between the electrodes. This ensures the highest possible conductivity, thereby guaranteeing operation of the cell within a temperature range of -40 °C to +80 °C.\autocite{Lith-Akku}
%Ein Elektrolyt ist im Wesentlichen eine Flüssigkeit, die mit Leitsalzen angereichert ist, um den Ionentransport zu ermöglichen. Darüber hinaus muss das Elektrolyt eine hohe Stabilität aufweisen, um mehreren tausend Lade- und Entladezyklen standzuhalten. 
An electrolyte is defined as a liquid that has been enriched with conductive salts to facilitate the movement of ions. Furthermore, the electrolyte must demonstrate exceptional stability in order to withstand numerous charge and discharge cycles, often numbering in the thousands.\autocite[S.61f.]{Korthauer.2013}
%Der Separator bildet eine Trennschicht innerhalb des Elektrolyten zwischen den Elektroden einer Lithium-Ionen-Zelle. Er besteht in der Regel aus einer Membran oder einem Vliesstoff aus Materialien wie Glasfaser oder Kunststoffen und weist eine Porosität von etwa 40 \% auf. Seine besondere Eigenschaft ist die selektive Durchlässigkeit für Ionen, die für die Umwandlung von chemischer in elektrische Energie unerlässlich sind. Elektronen hingegen werden durch den Separator blockiert, um sicherzustellen, dass sie über externe Leitungen zu den Verbrauchern, wie beispielsweise einem Steuergerät, transportiert werden können. Nach ihrer Nutzung kehren die Elektronen über den externen Stromkreis in die Zelle zurück, wo sie auf die gegenüberliegende Seite zu den Ionen gelangen. \newline 
The separator constitutes a separating layer within the electrolyte between the electrodes of a lithium-ion cell. The material under discussion is typically composed of a membrane or a non-woven fabric, which is manufactured using materials such as glass fibre or plastics. The material possesses a porosity of approximately 40\%. The subject's distinctive attribute is its selective permeability for ions, which are indispensable for the conversion of chemical energy into electrical energy. Conversely, electrons are obstructed by the separator, thereby ensuring their transportation to consumers, such as a control unit, via external lines. Following utilisation, the electrons return to the cell via the external circuit, where they reach the ions on the opposite side.\newline
The separator plays a crucial role in preventing internal short circuits that would occur without it. Furthermore, it has been demonstrated that the subject in question facilitates gas exchange by means of the absorption of the electrolyte. The physical characteristics of the separator, including its thickness and porosity, exert a substantial influence on the internal resistance of the cell. Consequently, these characteristics are pivotal in determining the overall performance of the system.\autocite[S. 80]{Korthauer.2013}
%Beim Laden einer Lithium-Ionen-Batterie fungiert die positive Elektrode als Anode, während sie beim Entladen als Kathode dient. Der Ladevorgang erfolgt üblicherweise nach dem sogenannten CC-CV-Verfahren (Constant Current - Constant Voltage). Dabei wird zunächst ein konstanter Strom (Constant Current, CC) angelegt, bis die Batterie eine festgelegte Spannung erreicht. Anschließend wird die Spannung konstant gehalten (Constant Voltage, CV), wobei der Stromfluss progressiv abnimmt. Die Beendigung des Ladevorgangs erfolgt in der Regel durch eine vorgegebene Zeitbegrenzung oder das Erreichen einer definierten Stromschwelle. 
In the context of lithium-ion battery operation, the positive electrode functions as the anode during the charging cycle and as the cathode during the discharging cycle. The charging process is typically executed through the utilisation of the Constant Current - Constant Voltage (CC-CV) method. In the initial step of the process, a constant current (CC) is applied until the battery has been stabilised at a fixed voltage. The voltage is then maintained at a constant level (constant voltage, CV), resulting in a progressive decrease in current flow. The charging process is typically terminated by the expiry of a stipulated time limit or when a predetermined current threshold is attained.\autocite[S. 15]{Korthauer.2013}
%Lithium-Ionen-Akkumulatoren sind stark temperaturabhängig, da bei sehr niedrigen Temperaturen der Innenwiderstand deutlich ansteigt. Dies ist auf die verlangsamten chemischen Reaktionen innerhalb der Zelle zurückzuführen. Darüber hinaus ist es essenziell, eine Überladung der Batterie zu vermeiden, da dies zu sogenannten Zerfallsreaktionen führen kann. Die Intensität dieser Reaktionen variiert je nach den verwendeten Materialien der Zellkomponenten und kann die Lebensdauer sowie die Sicherheit der Batterie erheblich beeinträchtigen. 
Lithium-ion batteries exhibit a substantial temperature dependency. It has been demonstrated that, at low temperatures, there is a substantial increase in internal resistance. This phenomenon can be attributed to the reduced rate of chemical reactions within the cell. Furthermore, it is imperative to refrain from overcharging the battery, as this can precipitate so-called decay reactions. The intensity of these reactions is contingent on the materials utilised for the cell components, with the potential to exert a substantial influence on the service life and safety of the battery.\autocite[S. 15f.]{Korthauer.2013}


\section{Technical Context of a Battery Enclosure}
Battery enclosures are critical components in the design of energy storage systems, particularly in mobile and compact applications. They serve as structural containers for lithium-ion cells and provide necessary protection against environmental and mechanical influences. Their design must reconcile several competing requirements: mechanical robustness, thermal management, electrical insulation, compactness, and manufacturability. With the increasing adoption of 3D printing technologies, engineers now have more flexibility to prototype and produce such enclosures, particularly for small-series or custom applications \cite{gebhardt2016}.

Modern lithium-ion cells come in standardized formats, such as cylindrical (e.g., 18650), prismatic, or pouch cells. Each of these formats imposes specific geometric and thermal constraints on the enclosure. Cylindrical cells, for example, are highly space-efficient in tightly packed arrays but require firm fixation and vibration dampening, as well as thermal spacing to avoid overheating. Enclosures for such cells often include integrated cell holders, structural ribs, and defined cooling pathways \cite{pistoia2018}. 

An essential consideration in battery pack design is thermal management. Since lithium-ion batteries are sensitive to excessive heat, the enclosure must ensure proper heat dissipation. Passive solutions, such as airflow channels or thermally conductive plastics, can be incorporated into the design. In high-performance applications, the enclosure may include embedded cooling elements. The thermal behavior of the entire assembly must be taken into account early in the design phase to avoid heat accumulation and ensure battery safety and longevity.

Mechanical constraints also play a decisive role. Battery packs are often subjected to vibration, shocks, and compression forces. Therefore, the housing material must be both strong and lightweight. Common materials used in 3D printing for such applications include ABS, PETG, and polyamide (PA12), each of which offers a specific balance of mechanical resilience, thermal resistance, and printability \cite{gebhardt2016}.

The enclosure design must also accommodate connectors, cable guides, ventilation openings, and possibly fasteners for mounting within a device or vehicle. All of these features must be precisely aligned and dimensioned to ensure a secure and reliable assembly. Such complexity makes parametric and constraint-based design software particularly valuable.

\section{Standards for drawing circuit symbols}

%\subsection*{Entstehung und Bedeutung von Normen}
%Normen haben ihren Ursprung in der industriellen Revolution, als der Bedarf an standardisierten Verfahren und Produkten exponentiell anstieg. Unterschiedliche Maße, Zeichnungen oder Bezeichnungen führten zu Missverständnissen, Ineffizienzen und Fehlern in der Fertigung und Kommunikation. Um diesem Chaos entgegenzuwirken, wurden Normen geschaffen, die als verbindliche Regelwerke dienen.
%
%Normen ermöglichen eine einheitliche Sprache zwischen Ingenieuren, Herstellern und Anwendern. Sie sichern die Kompatibilität von Bauteilen, verbessern die Qualität und fördern den internationalen Handel. Im Kontext technischer Zeichnungen – insbesondere von Schaltzeichen – gewährleisten Normen, dass technische Pläne weltweit eindeutig verstanden werden können, unabhängig von Sprache oder regionalen Besonderheiten.
The origins of standards can be traced back to the industrial revolution, a period which witnessed a significant increase in the demand for standardised processes and products. It is evident that discrepancies in dimensions, drawings and designations have resulted in a number of issues, including misunderstandings, inefficiencies and errors in production and communication. In order to combat this apparent chaos, a set of standards were established that now serve as legally binding regulations.\newline
Standards serve to facilitate a uniform language across the engineering, manufacturing and user communities. The primary functions of the components in question are threefold: firstly, to ensure compatibility; secondly, to improve quality; and thirdly, to promote international trade. In the context of technical drawings, particularly circuit symbols, standards play a pivotal role in ensuring the comprehension of technical plans on a global scale. This is achieved by transcending linguistic and regional barriers, thereby facilitating understanding irrespective of individual differences.
\subsection*{The most widely recognised standard symbols for circuit design}

%Drei der bekanntesten und am häufigsten verwendeten Normen für Schaltzeichen sind:
Three of the most widely recognised standart symbols for circut design are:
\begin{itemize}
%\item \textbf{DIN-Normen (Deutschland): }Diese Normen, herausgegeben vom Deutschen Institut für Normung, sind insbesondere im deutschsprachigen Raum verbreitet. Sie umfassen eine breite Palette von Standards, darunter auch solche für elektrische, hydraulische und pneumatische Schaltzeichen.
\item \textbf{DIN standards (Germany): }These standards, promulgated by the German Institute for Standardisation, are particularly widespread in German-speaking countries. The course material covers a broad spectrum of standards, encompassing those pertaining to electrical, hydraulic and pneumatic circuit symbols.
%\item \textbf{IEC-Normen (International):} Die International Electrotechnical Commission (IEC) ist für die Entwicklung global gültiger Standards verantwortlich. Die IEC 60617-Serie beispielsweise definiert Symbole für elektrotechnische Anlagen und Komponenten.
\item \textbf{IEC standards (International):} The International Electrotechnical Commission (IEC) is responsible for the development of standards that are applicable on a global scale. The IEC 60617 series, for instance, provides a standardized framework for the representation of symbols employed in the domain of electrotechnical systems and components.
%\item \textbf{ANSI-Normen (USA): }Das American National Standards Institute (ANSI) ist die dominierende Normierungsorganisation in den USA. ANSI-Zeichnungen sind häufig in nordamerikanischen Projekten anzutreffen.
\item \textbf{ANSI standards (USA):} The American National Standards Institute (ANSI) is the preeminent standardisation organisation in the USA. ANSI drawings are frequently encountered in North American projects.
\end{itemize}
%Die Wahl der Norm hängt von der Region und dem Anwendungsfall ab. Während europäische Projekte häufig auf DIN- oder IEC-Normen basieren, dominieren ANSI-Normen in den USA.
The choice of standard depends on the region and the application. While European projects are often based on DIN or IEC standards, ANSI standards dominate in the USA.

\subsection*{The DIN standard for circuit symbols in detail}

%Die DIN-Normen sind in Deutschland der zentrale Standard für die Erstellung technischer Zeichnungen und Schaltpläne. Besonders relevant ist die Norm DIN EN 60617, die elektrische Schaltzeichen beschreibt. Diese Norm wurde in Zusammenarbeit mit der IEC entwickelt, was die internationale Anschlussfähigkeit erleichtert.
In Germany, DIN standards play a pivotal role in the realm of technical drawing and circuit diagram creation. The DIN EN 60617 standard, which provides a comprehensive description of electrical circuit symbols, is of particular relevance in this context. The standard in question was developed in cooperation with the International Electrotechnical Commission (IEC), an international standards organisation that facilitates international connectivity.

DIN EN 60617 regulates in detail:

\begin{itemize}
%	\item \textbf{Die Darstellung von Bauelementen:} Elektronische Bauteile wie Widerstände, Kondensatoren oder Schalter haben klar definierte Symbole.
%	\item \textbf{Das Layout von Schaltplänen:} Vorgaben für Linienführung, Anschlussstellen und Abstände zwischen Symbolen sorgen für Übersichtlichkeit.
%	\item \textbf{Verbindungsleitungen:} Die Darstellung von Leitungen und Kreuzungen vermeidet Missverständnisse, beispielsweise durch eindeutige Markierungen bei Verbindungen.
\item \textbf{The representation of components:} Electronic components such as resistors, capacitors or switches have clearly defined symbols.
\item \textbf{The layout of circuit diagrams:} Specifications for line routing, connection points and distances between symbols ensure clarity.
\item \textbf{Connecting cables:} The visualisation of lines and crossings avoids misunderstandings, for example by clearly marking connections.
\end{itemize}
%Ein zentrales Ziel der DIN-Norm ist es, Komplexität zu reduzieren und eine intuitive Lesbarkeit zu fördern. Zusätzlich berücksichtigt die Norm auch neuere Technologien und Entwicklungen, wodurch sie immer wieder aktualisiert wird.
%
%Durch die Einhaltung der DIN-Norm können Ingenieure sicherstellen, dass ihre Schaltpläne sowohl in der eigenen Organisation als auch international korrekt interpretiert werden. Normen sind daher nicht nur ein Werkzeug der Standardisierung, sondern auch ein Mittel zur Qualitätssteigerung und zur Vereinfachung technischer Prozesse.
The DIN standard is predicated on the principle of reducing complexity and promoting intuitive readability. Moreover, the standard incorporates contemporary technologies and developments, necessitating regular updates to maintain currency.

Adherence to the DIN standard is of paramount importance for engineers seeking to ensure the accurate interpretation of their circuit diagrams within both their own organisation and on an international level. Standards can therefore be regarded not only as a tool for standardisation, but also as a means of improving quality and simplifying technical processes.
\section{Autodesk Fusion 360}
\label{Autodesk}
%Autodesk Fusion 360 ist eine integrierte Plattform für computergestütztes Design (CAD), Fertigung (Computer-Aided Engineering, kurz CAM) und technische Analyse (Computer-Aided Manufacturing, kurz CAE), die als Cloud-basierte Lösung entwickelt wurde. Sie erlaubt es, mechanische und elektronische Designprozesse zu vereinen, und bietet damit Ingenieuren, Designern und Entwicklern eine zentrale Plattform für die Produktentwicklung. Im Folgenden wird zunächst die Unternehmensgeschichte von Autodesk als Entwickler dieser Software beleuchtet, bevor die Kernfunktionen und speziellen Funktionen zur Erstellung elektronischer Schaltpläne detailliert werden.
Autodesk Fusion 360 is an integrated platform for computer-aided design (CAD), computer-aided engineering (CAM) and computer-aided manufacturing (CAE) that was developed as a cloud-based solution (Jones, 2019). The integration of mechanical and electronic design processes is facilitated, thereby providing engineers, designers and developers with a centralised platform for product development. The following discussion will firstly provide a brief overview of the company history of Autodesk, the developer of this software. Following this, the core and special functions for creating electronic circuit diagrams will be outlined in detail.

\subsection*{History and development}
%Autodesk Incorporated (Inc.) wurde 1982 von John Walker und einer Gruppe von Programmierern gegründet und spezialisierte sich schnell auf Softwarelösungen für Architektur, Ingenieurwesen und digitale Medien. 
Autodesk Incorporated (Inc.) was founded in 1982 by John Walker and a group of programmers and quickly specialised in software solutions for architecture, engineering and digital media \autocite{wikipedia_autodesk}.
%Die Veröffentlichung von AutoCAD im Jahr 1982 setzte einen wichtigen Meilenstein für die computergestützte Konstruktion und wurde zur führenden CAD-Software für Architekten und Ingenieure weltweit.
The release of AutoCAD in 1982 set an important milestone for computer-aided design and became the leading CAD software for architects and engineers worldwide\autocite{wikipedia_autocad_version_history}.


%Mit dem Aufkommen neuer Anforderungen in der Fertigungsindustrie und der Integration von Elektronik in mechanische Systeme begann Autodesk, eine neue Art von Software zu entwickeln. Ziel war es, die Mechanik- und Elektronikentwicklung auf einer Plattform zu vereinen und kollaboratives, Cloud-basiertes Arbeiten zu ermöglichen. Dies führte zur Einführung von Fusion 360 im Jahr 2013. 
The advent of novel requirements within the manufacturing industry, coupled with the integration of electronics into mechanical systems, prompted Autodesk to embark on the development of a novel software type. The objective of this initiative was to consolidate mechanical and electronic development on a unified platform, thereby facilitating collaborative, cloud-based operations. This development subsequently led to the introduction of Fusion 360 in 2013.\autocite{wikipedia_autodesk_deutsch}
%Durch die Integration traditioneller CAD/CAM/CAE-Funktionen und die cloudbasierte Zusammenarbeit wurde Fusion 360 zu einem beliebten Werkzeug in der Produktentwicklung und verhalf Autodesk zu einer neuen Marktposition im Bereich der digitalen Fertigung.
The integration of conventional CAD/CAM/CAE functions with cloud-based collaboration has resulted in the popularity of Fusion 360 as a tool in product development, thereby enabling Autodesk to achieve a novel market position in the domain of digital manufacturing.


\section{3D Design Approach Using Fusion 360}
Autodesk Fusion 360 provides an integrated environment for computer aided design (CAD), simulation, and Computer-Aided Manufacturing (CAM), which makes it particularly well-suited for iterative design and prototyping of technical components like battery enclosures. Its parametric modeling capabilities allow engineers to define the relationships between different parts of the geometry, ensuring that dimensional adjustments propagate automatically throughout the model \cite{hogan2025}.

Using Fusion 360, a designer can first define a master sketch that includes key dimensions such as cell spacing, wall thickness, and screw positions. Through extrusion and patterning, these base geometries are transformed into 3D solids. Additional features such as ventilation slots or mounting flanges can be added using derived sketches and Boolean operations\cite{hogan2025}..

Assemblies in Fusion 360 enable designers to position and constrain battery cells within the enclosure, simulating real-world configurations. This allows for spatial validation and interference checking early in the process, reducing the risk of design flaws during manufacturing\cite{hogan2025}..

Moreover, Fusion 360 supports exporting the final design directly into formats suitable for additive manufacturing, such as STL or 3MF. This integration streamlines the workflow from design to production, making it ideal for rapid prototyping and validation\cite{hogan2025}..

From a manufacturability perspective, the designer must also follow principles of Design for Additive Manufacturing (DfAM). This includes minimizing unsupported overhangs, ensuring even wall thicknesses, and aligning features for optimal layer orientation. Fusion 360 offers visualization tools and slicer integration to help evaluate the printability of the part \cite{anderson2020}.

In summary, Fusion 360 offers the necessary flexibility and functionality to develop complex battery enclosures that meet structural, thermal, and electrical requirements. Its parametric design environment, combined with visualization and export tools, makes it an effective platform for realizing functional prototypes that are ready for testing and refinement.

