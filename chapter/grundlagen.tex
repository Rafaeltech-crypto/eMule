\chapter{Grundlagen}
\label{cha:Grundlagen}
	\begin{itemize}
	\item Normen -> Din norm
	\item wenn man strecken muss: kawasaki mule+definition elektrofzg
\end{itemize}
\section{Normen zur Zeichnung von Schaltzeichen}

\subsection*{Entstehung und Bedeutung von Normen}

Normen haben ihren Ursprung in der industriellen Revolution, als der Bedarf an standardisierten Verfahren und Produkten exponentiell anstieg. Unterschiedliche Maße, Zeichnungen oder Bezeichnungen führten zu Missverständnissen, Ineffizienzen und Fehlern in der Fertigung und Kommunikation. Um diesem Chaos entgegenzuwirken, wurden Normen geschaffen, die als verbindliche Regelwerke dienen.

Normen ermöglichen eine einheitliche Sprache zwischen Ingenieuren, Herstellern und Anwendern. Sie sichern die Kompatibilität von Bauteilen, verbessern die Qualität und fördern den internationalen Handel. Im Kontext technischer Zeichnungen – insbesondere von Schaltzeichen – gewährleisten Normen, dass technische Pläne weltweit eindeutig verstanden werden können, unabhängig von Sprache oder regionalen Besonderheiten.

\subsection*{Die bekanntesten Normen für Schaltzeichen}

Drei der bekanntesten und am häufigsten verwendeten Normen für Schaltzeichen sind:
\begin{itemize}
\item \textbf{DIN-Normen (Deutschland): }Diese Normen, herausgegeben vom Deutschen Institut für Normung, sind insbesondere im deutschsprachigen Raum verbreitet. Sie umfassen eine breite Palette von Standards, darunter auch solche für elektrische, hydraulische und pneumatische Schaltzeichen.
\item \textbf{IEC-Normen (International):} Die International Electrotechnical Commission (IEC) ist für die Entwicklung global gültiger Standards verantwortlich. Die IEC 60617-Serie beispielsweise definiert Symbole für elektrotechnische Anlagen und Komponenten.
\item \textbf{ANSI-Normen (USA): }Das American National Standards Institute (ANSI) ist die dominierende Normierungsorganisation in den USA. ANSI-Zeichnungen sind häufig in nordamerikanischen Projekten anzutreffen.
\end{itemize}
Die Wahl der Norm hängt von der Region und dem Anwendungsfall ab. Während europäische Projekte häufig auf DIN- oder IEC-Normen basieren, dominieren ANSI-Normen in den USA.

\subsection*{Die DIN-Norm für Schaltzeichen im Detail}

Die DIN-Normen sind in Deutschland der zentrale Standard für die Erstellung technischer Zeichnungen und Schaltpläne. Besonders relevant ist die Norm DIN EN 60617, die elektrische Schaltzeichen beschreibt. Diese Norm wurde in Zusammenarbeit mit der IEC entwickelt, was die internationale Anschlussfähigkeit erleichtert.

Die DIN EN 60617 regelt detailliert:
\begin{itemize}
	\item \textbf{Die Darstellung von Bauelementen:} Elektronische Bauteile wie Widerstände, Kondensatoren oder Schalter haben klar definierte Symbole.
	\item \textbf{Das Layout von Schaltplänen:} Vorgaben für Linienführung, Anschlussstellen und Abstände zwischen Symbolen sorgen für Übersichtlichkeit.
	\item \textbf{Verbindungsleitungen:} Die Darstellung von Leitungen und Kreuzungen vermeidet Missverständnisse, beispielsweise durch eindeutige Markierungen bei Verbindungen.
\end{itemize}
Ein zentrales Ziel der DIN-Norm ist es, Komplexität zu reduzieren und eine intuitive Lesbarkeit zu fördern. Zusätzlich berücksichtigt die Norm auch neuere Technologien und Entwicklungen, wodurch sie immer wieder aktualisiert wird.

Durch die Einhaltung der DIN-Norm können Ingenieure sicherstellen, dass ihre Schaltpläne sowohl in der eigenen Organisation als auch international korrekt interpretiert werden. Normen sind daher nicht nur ein Werkzeug der Standardisierung, sondern auch ein Mittel zur Qualitätssteigerung und zur Vereinfachung technischer Prozesse.

\section{Autodesk Fusion 360}
\label{Autodesk}
Autodesk Fusion 360 ist eine integrierte Plattform für computergestütztes Design (CAD), Fertigung (CAM) und technische Analyse (CAE), die als Cloud-basierte Lösung entwickelt wurde. Sie erlaubt es, mechanische und elektronische Designprozesse zu vereinen, und bietet damit Ingenieuren, Designern und Entwicklern eine zentrale Plattform für die Produktentwicklung. Im Folgenden wird zunächst die Unternehmensgeschichte von Autodesk als Entwickler dieser Software beleuchtet, bevor die Kernfunktionen und speziellen Funktionen zur Erstellung elektronischer Schaltpläne detailliert werden.
\subsection{Installationsaleitung}
Anleitung zur Erstellung eines Studentenaccounts und zum Herunterladen von Fusion 360 Electronics.

\subsection*{Erstellung eines Autodesk-Studentenaccounts}

Zur Nutzung von Fusion 360 Electronics ist die Erstellung eines Autodesk-Studentenaccounts erforderlich. Dies ermöglicht den kostenlosen Zugriff auf die Software.

\paragraph{ Registrierung}
\begin{itemize}
	\item Zugriff auf die Registrierungsseite: \href{https://accounts.autodesk.com/register?resume=/as/fMRyxxIM12/resume/as/authorization.ping&ack=uWlmiJuqQqVaAQjGdojc8Qxit4KVdorZ}{\underline{Autodesk Registrierungsseite}}.
	\item Ausfüllen des Formulars mit den notwendigen Informationen:
	\begin{itemize}
		\item Vor- und Nachname
		\item Gültige E-Mail-Adresse
		\item Passwort entsprechend den Sicherheitsrichtlinien
	\end{itemize}
\end{itemize}

\paragraph{ Bestätigung der E-Mail-Adresse}
\begin{itemize}
	\item Nach dem Absenden des Formulars wird eine E-Mail zur Bestätigung empfangen.
	\item Öffnen der E-Mail und Klicken auf den Bestätigungslink zur Verifizierung der Adresse.
\end{itemize}

\paragraph{ Vervollständigung der Profilinformationen}
\begin{itemize}
	\item Anmeldung im Autodesk-Konto.
	\item Angabe weiterer Informationen wie Institution, Studienrichtung und Studienjahr zur Bestätigung des Studentenstatus.
\end{itemize}

\paragraph*{ Verifizierung des Studentenstatus}
\begin{itemize}
	\item Hochladen eines Dokuments, das die Immatrikulation belegt (z. B. eine Studienbescheinigung).
	\item Autodesk prüft die Dokumente innerhalb weniger Tage und sendet eine Bestätigung per E-Mail.
\end{itemize}

\subsection*{Herunterladen und Installieren von Fusion 360 Electronics}

\paragraph{Zugriff auf den Download-Bereich}
\begin{itemize}
	\item Nach erfolgreicher Verifizierung des Accounts erfolgt die Anmeldung und Navigation zur \href{https://www.autodesk.com/education/home}{\underline{Autodesk Education Community}}.
	\item Auswahl von Fusion 360 aus der Liste der verfügbaren Software.
\end{itemize}

\paragraph{ Download und Installation}
\subsection*{Windows}
\begin{itemize}
	\item Beachten Sie bei der Auswahl der Downloaddatei die Unterschiede zwischen den Softwareversionen für die verschiedenen Windows-Betriebssysteme. Diese unterscheiden sich in der Versionsnummer (z. B. \glqq Windows 11\grqq {}) und in den Bit-Versionen (32- und 64-Bit).
	\item Schritte zur Identifikation der Windows-Version:
		\begin{itemize}
		\item[1.] Drücke die Tastenkombination Windows-Taste + I, um die Einstellungen zu öffnen.
		\item[2.] Gehe zu System → Info.
		\item[3.] Unter Windows-Spezifikationen findest du die genaue Version und Edition von Windows (z. B. \glqq Windows 11 Pro\grqq {}, \glqq Version 22H2\grqq {}).
		\end{itemize}
	\item Schritte zur Identifikation der Bit-Version:
		\begin{itemize}
		\item[1.] Drücke die Tastenkombination Windows-Taste + I, um die Einstellungen zu öffnen.
		\item[2.] Gehe zu System → Info.
		\item[3.] 	Unter Gerätespezifikationen → Systemtyp steht z. B. \glqq 64-Bit-Betriebssystem\grqq {}.
		\end{itemize}
	\item Klicken auf „Jetzt herunterladen“ und Befolgen der Anweisungen auf dem Bildschirm.
	\item Nach Abschluss des Downloads Öffnen der Installationsdatei und Befolgen der Installationsanweisungen.
\end{itemize}
\subsection*{macOS}
\begin{itemize}
	\item Beachten Sie bei der Auswahl der Downloaddatei die Unterschiede zwischen der Softwareversion für Betriebssysteme mit Apple Silicon Prozessor und Intel Prozessor.
	\item Schritte zur Identifikation des verbauten Prozessors:
	\begin{itemize}
		\item[1.] Klicke oben links auf das Apple-Symbol.
		\item[2.] Wähle "Über diesen Mac".
		\item[3.] Schaue im Fenster, das sich öffnet: \newline
		Wenn dort \glqq Chip\grqq {} steht, gefolgt von z. B. \glqq Apple M1\grqq {} oder \glqq Apple M2\grqq {}, ist ein Apple Silicon Prozessor verbaut.\newline
		Wenn dort \glqq Prozessor\grqq {} steht, gefolgt von einem Intel-Prozessor (z. B. \glqq Intel Core i5\grqq {}), ist ein Intel-Prozessor in dem Mac verbaut.
	\end{itemize}
	\item Klicken auf „Jetzt herunterladen“ und Befolgen der Anweisungen auf dem Bildschirm.
	\item Nach Abschluss des Downloads Öffnen der Installationsdatei und Befolgen der Installationsanweisungen.
\end{itemize}

\paragraph*{Aktivierung der Education-Lizenz}
\begin{itemize}
	\item Beim ersten Start von Fusion 360 erfolgt die Eingabe der Anmeldeinformationen.
	\item Die Software erkennt automatisch den Studentenstatus und aktiviert die entsprechende Lizenz.
\end{itemize}





\subsection{Historie und Entwicklung}
Autodesk, Inc. wurde 1982 von John Walker und einer Gruppe von Programmierern gegründet und spezialisierte sich schnell auf Softwarelösungen für Architektur, Ingenieurwesen und digitale Medien. \autocite{wikipedia_autodesk}
Die Veröffentlichung von AutoCAD im Jahr 1982 setzte einen wichtigen Meilenstein für die computergestützte Konstruktion und wurde zur führenden CAD-Software für Architekten und Ingenieure weltweit.\autocite{wikipedia_autocad_version_history}


Mit dem Aufkommen neuer Anforderungen in der Fertigungsindustrie und der Integration von Elektronik in mechanische Systeme begann Autodesk, eine neue Art von Software zu entwickeln. Ziel war es, die Mechanik- und Elektronikentwicklung auf einer Plattform zu vereinen und kollaboratives, Cloud-basiertes Arbeiten zu ermöglichen. Dies führte zur Einführung von Fusion 360 im Jahr 2013. \autocite{wikipedia_autodesk_deutsch}
Durch die Integration traditioneller CAD/CAM/CAE-Funktionen und die cloudbasierte Zusammenarbeit wurde Fusion 360 zu einem beliebten Werkzeug in der Produktentwicklung und verhalf Autodesk zu einer neuen Marktposition im Bereich der digitalen Fertigung.
\subsection{Grundfunktionen}
\subsection{Spezielle Funktionen zur Erstellung von Stromlaufplänen und Bestückungsplänen}


Zielgerichtete theoretische Grundlagen, sowohl fachliche, wie auch methodische.

Zu den Grundlagen gehören z.~B. auch Details zur Problemstellung, der Stand der Technik und weitere Grundlagen, welche zur Konzeptausarbeitung, Umsetzung und Verifikation erforderlich sind.

Grundlagen haben immer einen Bezug zu den nachfolgenden Kapiteln. Diesen Bezug sollte man gelegentlich explizit herstellen, damit bereits in diesem Kapitel klar ist, wo und für was die Grundlagen gebraucht und angewandt werden.