\chapter{Grundlagen}
\label{cha:Grundlagen}

	\begin{itemize}
		\item Normen -> Din norm
		\item Autodesk
		\item wenn man strecken muss: kawasaki mule+definition elektrofzg
		\end{itemize}
Zielgerichtete theoretische Grundlagen, sowohl fachliche, wie auch methodische.

Zu den Grundlagen gehören z.~B. auch Details zur Problemstellung, der Stand der Technik und weitere Grundlagen, welche zur Konzeptausarbeitung, Umsetzung und Verifikation erforderlich sind.

Grundlagen haben immer einen Bezug zu den nachfolgenden Kapiteln. Diesen Bezug sollte man gelegentlich explizit herstellen, damit bereits in diesem Kapitel klar ist, wo und für was die Grundlagen gebraucht und angewandt werden.