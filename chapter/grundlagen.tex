\chapter{Grundlagen}
\label{cha:Grundlagen}
	\begin{itemize}
	\item Normen -> Din norm
	\item wenn man strecken muss: kawasaki mule+definition elektrofzg
\end{itemize}
\section{Autodesk Fusion 360
\label{Autodesk}
Autodesk Fusion 360 ist eine integrierte Plattform für computergestütztes Design (CAD), Fertigung (CAM) und technische Analyse (CAE), die als Cloud-basierte Lösung entwickelt wurde. Sie erlaubt es, mechanische und elektronische Designprozesse zu vereinen, und bietet damit Ingenieuren, Designern und Entwicklern eine zentrale Plattform für die Produktentwicklung. Im Folgenden wird zunächst die Unternehmensgeschichte von Autodesk als Entwickler dieser Software beleuchtet, bevor die Kernfunktionen und speziellen Funktionen zur Erstellung elektronischer Schaltpläne detailliert werden.
\subsection{Installationsaleitung}
Anleitung zur Erstellung eines Studentenaccounts und zum Herunterladen von Fusion 360 Electronics

\paragraph*{Erstellung eines Autodesk-Studentenaccounts}
Zur Nutzung von Fusion 360 Electronics ist die Erstellung eines Autodesk-Studentenaccounts erforderlich. Dies ermöglicht den kostenlosen Zugriff auf die Software.

\paragraph{ Registrierung}
\begin{itemize}
	\item Zugriff auf die Registrierungsseite: \href{https://accounts.autodesk.com/register?resume=/as/fMRyxxIM12/resume/as/authorization.ping&ack=uWlmiJuqQqVaAQjGdojc8Qxit4KVdorZ}{\underline{Autodesk Registrierungsseite}}.
	\item Ausfüllen des Formulars mit den notwendigen Informationen:
	\begin{itemize}
		\item Vor- und Nachname
		\item Gültige E-Mail-Adresse
		\item Passwort entsprechend den Sicherheitsrichtlinien
	\end{itemize}
\end{itemize}

\paragraph{ Bestätigung der E-Mail-Adresse}
\begin{itemize}
	\item Nach dem Absenden des Formulars wird eine E-Mail zur Bestätigung empfangen.
	\item Öffnen der E-Mail und Klicken auf den Bestätigungslink zur Verifizierung der Adresse.
\end{itemize}

\paragraph{ Vervollständigung der Profilinformationen}
\begin{itemize}
	\item Anmeldung im Autodesk-Konto.
	\item Angabe weiterer Informationen wie Institution, Studienrichtung und Studienjahr zur Bestätigung des Studentenstatus.
\end{itemize}

\paragraph*{ Verifizierung des Studentenstatus}
\begin{itemize}
	\item Hochladen eines Dokuments, das die Immatrikulation belegt (z. B. eine Studienbescheinigung).
	\item Autodesk prüft die Dokumente innerhalb weniger Tage und sendet eine Bestätigung per E-Mail.
\end{itemize}

\paragraph*{ Herunterladen und Installieren von Fusion 360 Electronics}

\paragraph{Zugriff auf den Download-Bereich}
\begin{itemize}
	\item Nach erfolgreicher Verifizierung des Accounts erfolgt die Anmeldung und Navigation zur \href{https://www.autodesk.com/education/home}{\underline{Autodesk Education Community}}.
	\item Auswahl von Fusion 360 aus der Liste der verfügbaren Software.
\end{itemize}

\paragraph{ Download und Installation}
\begin{itemize}
	\item Klicken auf „Jetzt herunterladen“ und Befolgen der Anweisungen auf dem Bildschirm.
	\item Nach Abschluss des Downloads Öffnen der Installationsdatei und Befolgen der Installationsanweisungen.
\end{itemize}

\paragraph*{Aktivierung der Education-Lizenz}
\begin{itemize}
	\item Beim ersten Start von Fusion 360 erfolgt die Eingabe der Anmeldeinformationen.
	\item Die Software erkennt automatisch den Studentenstatus und aktiviert die entsprechende Lizenz.
\end{itemize}




\subsection*{Windows}
\subsection*{Mac}
\subsection{Historie und Entwicklung}
Autodesk, Inc. wurde 1982 von John Walker und einer Gruppe von Programmierern gegründet und spezialisierte sich schnell auf Softwarelösungen für Architektur, Ingenieurwesen und digitale Medien. \autocite{wikipedia_autodesk}
Die Veröffentlichung von AutoCAD im Jahr 1982 setzte einen wichtigen Meilenstein für die computergestützte Konstruktion und wurde zur führenden CAD-Software für Architekten und Ingenieure weltweit.\autocite{wikipedia_autocad_version_history}


Mit dem Aufkommen neuer Anforderungen in der Fertigungsindustrie und der Integration von Elektronik in mechanische Systeme begann Autodesk, eine neue Art von Software zu entwickeln. Ziel war es, die Mechanik- und Elektronikentwicklung auf einer Plattform zu vereinen und kollaboratives, Cloud-basiertes Arbeiten zu ermöglichen. Dies führte zur Einführung von Fusion 360 im Jahr 2013. \autocite{wikipedia_autodesk_deutsch}
Durch die Integration traditioneller CAD/CAM/CAE-Funktionen und die cloudbasierte Zusammenarbeit wurde Fusion 360 zu einem beliebten Werkzeug in der Produktentwicklung und verhalf Autodesk zu einer neuen Marktposition im Bereich der digitalen Fertigung.
\subsection{Grundfunktionen}
Autodesk Fusion 360 ist eine umfassende Lösung, die verschiedene Aspekte der Produktentwicklung unterstützt und eine Vielzahl an Design- und Fertigungswerkzeugen bietet:
\subsubsection*{3D-Modellierung }
Fusion 360 bietet verschiedene Modellierungswerkzeuge für die parametrische Modellierung, Freiform-Modellierung und direkte Modellierung, die für das Design von mechanischen Komponenten bis hin zu organischen Formen verwendet werden können. Durch die parametrische Modellierung können Designer Abmessungen und Beziehungen zwischen Teilen präzise definieren und nachträglich anpassen.\autocite{autodesk_fusion360_features}
\subsubsection*{Simulation und Analyse}
Um die Festigkeit und Belastbarkeit von Bauteilen zu überprüfen, bietet Fusion 360 Simulationswerkzeuge für die Finite-Elemente-Analyse (FEA). Außerdem sind thermische und mechanische Simulationen integriert, die es ermöglichen, die Eigenschaften eines Produkts unter verschiedenen Bedingungen zu testen.\autocite{autodesk_what_is_fusion360}
\subsubsection*{Rendering}
Fusion 360 enthält Rendering-Werkzeuge, die photorealistische Darstellungen der entworfenen Modelle erzeugen. Diese Funktion ermöglicht es, das Design visuell zu präsentieren und potenzielle Änderungen frühzeitig zu erkennen.\autocite{autodesk_fusion360_lessons}
\subsubsection*{Generative Gestaltung}
Diese Funktion nutzt künstliche Intelligenz, um automatisch optimierte Designalternativen zu erzeugen. Unter Berücksichtigung von Konstruktionszielen wie Materialeinsparung und Stabilität werden alternative Formen erstellt, die für strukturelle Belastungen optimiert sind.\autocite{cideon_fusion360_vorteile}
\subsubsection*{Fertigung und CAM-Werkzeuge}
Fusion 360 verfügt über umfangreiche CAM-Funktionalitäten, die CNC-Programmierung, 3D-Druck und andere Fertigungsprozesse unterstützen. Entwickler können G-Code für Maschinen erstellen und so den Übergang vom digitalen Design zur physischen Produktion nahtlos gestalten.\autocite{autodesk_fusion360_lessons}
\subsubsection*{Cloud-basierte Zusammenarbeit}
Fusion 360 speichert Projekte in der Cloud und ermöglicht damit eine Echtzeit-Zusammenarbeit und Versionskontrolle. Teams können von unterschiedlichen Standorten auf die Projekte zugreifen und Änderungen unmittelbar teilen, was die Entwicklung beschleunigt und den Austausch zwischen verschiedenen Disziplinen erleichtert.\autocite{cideon_fusion360_vorteile}
\subsection{Spezielle Funktionen zur Erstellung von Schaltplänen, Bestückungsplänen und Stromlaufplänen}


Zielgerichtete theoretische Grundlagen, sowohl fachliche, wie auch methodische.

Zu den Grundlagen gehören z.~B. auch Details zur Problemstellung, der Stand der Technik und weitere Grundlagen, welche zur Konzeptausarbeitung, Umsetzung und Verifikation erforderlich sind.

Grundlagen haben immer einen Bezug zu den nachfolgenden Kapiteln. Diesen Bezug sollte man gelegentlich explizit herstellen, damit bereits in diesem Kapitel klar ist, wo und für was die Grundlagen gebraucht und angewandt werden.