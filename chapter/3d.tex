\chapter{3D Modelling}
The design of battery housings is an essential step in developing reliable energy storage systems. Autodesk Fusion 360 offers an integrated CAD environment ideal for creating precise and adaptable models, especially when targeting additive manufacturing methods. This section focuses on the detailed process of creating, structuring, and refining battery enclosure CAD files within Fusion 360, emphasizing parametric design and manufacturability \cite{hogan2025}.
\section{Design and Development of a Modular Lithium-Ion Battery System}

The design illustrates a modular lithium-ion battery storage system, developed for high-density energy applications such as stationary energy storage, backup infrastructure, or mobile electrification platforms. The system is based on 21700-format lithium-ion cells, arranged in a densely packed structural configuration to optimize both volumetric energy density and thermal dissipation properties. The design process was carried out using CAD modeling tools, including parametric 3D assemblies, with iterative thermal and structural simulations executed to ensure mechanical and thermal stability under nominal and elevated load conditions.



\section{Parametric Modeling Principles}

Parametric modeling underpins efficient design workflows in Fusion 360. By defining key dimensions and constraints as parameters, changes can be propagated throughout the model automatically. This approach enables rapid iteration and consistent accuracy across complex geometries \cite{anderson2020}.

Initial modeling begins with sketches representing cross-sectional profiles of the battery cells and enclosure features. These sketches are converted into 3D geometry via extrusion, revolution, or other solid modeling operations. Parameters controlling dimensions such as cell diameter, wall thickness, and sensor pocket size are set at the outset, allowing easy adjustments during development \cite{gebhardt2016}.

\section{Base Geometry and Cell Arrangement}

The first step is designing the footprint to hold the battery cells securely. For cylindrical 18650 cells, a circular sketch with a diameter of 18.6 mm is created. To accommodate tolerances and thermal expansion, an additional clearance of approximately 1.5 to 2 mm is added around each cell \cite{pistoia2018}. Using Fusion 360’s pattern features, this base slot is duplicated to form the desired cell matrix arrangement.

Next, the outline of the overall enclosure is sketched, defining the external boundaries. The sketch is extruded to form the base structure, with wall thickness set according to mechanical strength requirements and printability constraints, typically ranging from 2 mm to 4 mm \cite{gebhardt2016}.

\section{Feature Addition: Sensor Integration and Cable Management}

A critical aspect of the enclosure design is the integration of temperature sensors. Dedicated pockets are created by cutting recesses adjacent to the battery cells using parametric cut operations. These pockets are dimensioned to fit common thermistors or digital sensors securely, ensuring reliable thermal contact \cite{anderson2020}.

Additionally, routing channels for sensor wiring are incorporated into the design. These channels prevent wire interference and facilitate clean assembly. All such features are parameter-driven, allowing dimension adjustments as sensor specifications or wiring requirements evolve.

\section{Assembly Modeling and Component Placement}

Fusion 360’s assembly workspace enables importing battery cell and sensor component models. These are positioned within the enclosure to verify fit, clearances, and spatial relationships. The assembly environment supports the use of constraints and joints, allowing realistic simulation of component interactions \cite{hogan2025}.

Parameters controlling cell spacing, sensor pocket depth, and cable routing pathways are globally defined. Altering any parameter updates the entire assembly model, significantly improving design iteration speed and accuracy.

\section{Design for Additive Manufacturing (DfAM) Considerations}

Throughout the modeling process, adherence to DfAM principles is crucial to ensure manufacturability via 3D printing. This includes minimizing overhangs exceeding recommended angles (usually 45°), maintaining uniform wall thickness to prevent warping, and designing features to reduce the need for support material \cite{anderson2020}.

Fillets and chamfers are strategically applied to edges to improve mechanical strength and printing quality. The model is split into modular subassemblies where necessary, facilitating multi-part printing and post-processing.

\section{Manufacturing Preparation and Export}

The final step in the design workflow involves exporting the CAD files into formats compatible with slicing software, primarily STL or 3MF \cite{gebhardt2016}. Fusion 360’s built-in manufacturing tools allow users to simulate print jobs, preview layer-by-layer builds, and optimize part orientation for strength and surface finish.

Subsequent iterations incorporate feedback from prototype prints, with dimension adjustments made via the parametric model to fine-tune fit and function.

\section{Iterative Refinement and Version Control}

Due to the parametric nature of the model, any necessary refinements—such as modifying sensor pocket sizes or wall thickness—are efficiently implemented without reconstructing the design from scratch \cite{hogan2025}. Fusion 360’s version control system aids in tracking changes and managing design variants.

This iterative design methodology accelerates development timelines and reduces errors, particularly when integrating complex features like sensor integration and wiring management.



\section{Application Context: Battery Housing for the E-Mule Energy Storage System}
\includepdf[pages=1, fitpaper=true, pagecommand={%
	\thispagestyle{empty} % Entfernt Seitenzahlen und Kopfzeilen
	\begin{center}
		\captionof{figure}{etailed 3D CAD rendering of a lithium-ion cell used in the energy storage system.} % Fügt die Caption ein
		\label{fig:Zelle} % Für Querverweise
	\end{center}
}]{circuts/Zelle.pdf}
\addtocounter{page}{1} % Seitenzähler korrekt erhöhen

The battery pack presented in \ref{fig:Zelle}  was designed specifically for use in an electric utility vehicle—the E-Mule. In such mobile applications, energy storage must not only provide sufficient capacity and power density, but also fulfill mechanical and thermal requirements, ensure serviceability, and allow compact packaging. These constraints played a central role in the CAD modeling process and directly influenced key design decisions.

The final design was entirely created using Fusion 360, with a strong focus on modularity, manufacturability, and the physical integration of standard lithium-ion cells (type 21700). The enclosure and internal structures were tailored for additive manufacturing using PETG, a common thermoplastic with suitable strength and heat resistance \cite{gebhardt2016}. No electrical testing or thermal simulation was conducted as part of this stage; the focus remained on the mechanical layout and enclosure architecture.

\section{Detailed CAD Design Process in Fusion 360}

\subsection{Cell Holder Design and Arrangement}

The first modeling step was the design of an individual battery cell holder (\ref{fig:Zelle}). Each cell is a cylindrical 21700 lithium-ion battery, typically 21 mm in diameter and 70 mm in length. In Fusion 360, a 2D sketch was created with circular cutouts for each cell, spaced evenly in a grid pattern. These cutouts were then extruded to form vertical cavities, which securely hold the cells while leaving sufficient clearance for thermal expansion and wiring.

To ensure consistent wall thickness and clearance, the design used parameterized dimensions tied to a master sketch. This allowed rapid iteration and adjustment of the number of cells in the matrix. 

The resulting layout provides space for a total of 70 cells (14× 5 array), which was deemed sufficient for the E-Mule’s use case in terms of energy content. The precision and symmetry of the layout were maintained through pattern tools and the use of midplane construction lines.

% Assembly mit Gehäuse
\includepdf[pages=1, fitpaper=true, pagecommand={%
	\thispagestyle{empty}
	\begin{center}
		\captionof{figure}{Fully assembled energy storage module.}
		\label{fig:Assembly}
	\end{center}
}]{circuts/Energiespeichersystem_Assembley_TOP.pdf}
\addtocounter{page}{1}

After creating the single cell holder as a component, it was duplicated and arranged in Fusion 360’s assembly environment to simulate the full module layout, seen in \ref{fig:Assembly}. This stage focused on integrating all components—holders, cell blocks, connector slots, sensor openings—into a complete mechanical structure.


During assembly, care was taken to maintain access to each row for both cooling and cabling. Clearances were checked using section analysis and interference detection tools within Fusion 360. Dummy models of power connectors and temperature sensors were placed to simulate real-world installation.



\subsection{Integration of the Protective Enclosure}
\includepdf[pages=1, fitpaper=true, pagecommand={%
	\thispagestyle{empty}
	\begin{center}
		\captionof{figure}{Fully assembled energy storage module with housing for use in the eMule vehicle.}
		\label{fig:AssemblyHousing}
	\end{center}
}]{circuts/Energiespeichersystem_Assembley_TOP_mit gehause.pdf}
\addtocounter{page}{1}
The final design step was the addition of a functional enclosure as seen in \ref{fig:AssemblyHousing}. The enclosure was modeled as a single shell body with integrated ventilation openings, flanged screw points, and access cutouts for connectors. Its purpose is to protect the cells from mechanical impact, environmental contamination, and unintentional contact.

The enclosure follows the contour of the internal components closely, minimizing unused space while allowing air to circulate between components. To accommodate additive manufacturing constraints, all overhangs were designed with a maximum angle of 45°, and fillets were added at all interior corners to prevent stress concentration. A removable lid was included to enable servicing of the battery.

The figure seen in  \ref{fig:AssemblyHousing} illustrates the fully enclosed battery system, ready for 3D printing.

% Assembly mit Gehäuse




