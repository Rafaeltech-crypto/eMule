\chapter{Zusammenfassung}
\label{cha:zusammenfassung}

Auf zwei bis drei Seiten soll auf folgende Punkte eingegangen werden:

\begin{itemize}
	\item Welches Ziel sollte erreicht werden
	\item Welches Vorgehen wurde gewählt
	\item Was wurde erreicht, zentrale Ergebnisse nennen, am besten quantitative Angaben machen
	\item Konnten die Ergebnisse nach kritischer Bewertung zum Erreichen des Ziels oder zur Problemlösung beitragen
	\item  Ausblick
\end{itemize}

In der Zusammenfassung sind unbedingt klare Aussagen zum Ergebnis der Arbeit zu nennen. Üblicherweise können Ergebnisse nicht nur qualitativ, sondern auch quantitativ benannt werden, z.~B. \glqq \ldots konnte eine Effizienzsteigerung von \SI{12}{\percent} erreicht werden.\grqq~oder \glqq \ldots konnte die Prüfdauer um \SI{2}{\hour} verkürzt werden\grqq.

Die Ergebnisse in der Zusammenfassung sollten selbstverständlich einen Bezug zu den in der Einleitung aufgeführten Fragestellungen und Zielen haben.
\section{Ausblick}
Ergänzend zu der nach aktuellem Stand vorhanden Dokumentation sollen im nächsten Semester weitere Stromlaufpläne generiert und in diese aufgenommen werden. Ergänzend sollen der Dokumentation Bestückungspläne hinzugefügt werden. Die gesamte Dokumentation soll in einem großes Dokument als ein Projekt mit sämtlichen Projekttiefen festgehalten werden. Weitere Ziele für das nächste Semester umfassen eine Modularisierung der Batterie, den Einbau des Bussystems an die Batterie, eine Erweiterung der Temeraturüberwachung sowie die Modularisierung der Powerbox und die Installation eines Displays im Cockpit, welches sämtliche Fahr-, Verbrauchs-, und Leistungsdaten anzeigt. Als optionales Ziel soll bei entsprechendem Fortschritt des Systems noch eine Musikanlage verbaut werden.
Modularisierung der Batterie, (batterieoptimierung(z.b. Layout)), Bussystem an der Batterie 
