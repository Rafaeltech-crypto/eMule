\chapter{Kritische Reflektion und Ausblick}
\label{cha:zusammenfassung}

%Auf zwei bis drei Seiten soll auf folgende Punkte eingegangen werden:

Im Rahmen des Projekts eMule 7.0 wurden verschiedene Ziele für das Wintersemester 2024 definiert, darunter:  
\begin{itemize}
	\item[1.] Fortsetzung der Entwicklung und Installation des neuen Energiespeichersystems in Form einer Traktionsbatterie für das eMule-Fahrzeug,
	\item[2.] Weiterentwicklung des Kühlsystems der Traktionsbatterie,
	\item[3.] Verbesserung der Temperaturregelung des Kühlsystems durch Arduino-Mikrocontrollerprogrammierung,
	\item[4.] Optimierung der Verkabelung des Traktionsbatteriemanagementsystems,
	\item[5.] Integration zusätzlicher Sensoren,
	\item[6.] Erstellung aktueller Stromlauf- und Bestückungspläne der \newline elektrisch/elektronischen Fahrzeugschaltkreise,
	\item[7.] Standardisierung und Vereinheitlichung der System- und Einzelkomponentenpläne,
	\item[8.] Entwicklung einer Installationsanleitung für Autodesk Fusion 360 für verschiedene Betriebssysteme,
	\item[9.] Durchführung von Messungen und Funktionstests,
	\item[10.] Vorbereitung des Fahrzeugs auf die bevorstehende Elektromagnetische Verträglichkeit (EMV) Prüfung und Technischer Überwachungsverein (TÜV) Abnahme.
\end{itemize}

Da sich diese Arbeit ausschließlich auf die Ziele 6 bis 8 konzentriert, werden im Folgenden nur diese detailliert betrachtet.  

Um die Erstellung aktueller Stromlaufpläne effizient zu gestalten, wurden zwei verschiedene Vorgehensweisen angewendet. Bei der Aktualisierung bereits vorhandener Pläne wurde folgender Prozess implementiert:  
Die bestehenden Pläne wurden zunächst ausgedruckt und systematisch auf Unstimmigkeiten, wie fehlende Verbindungen oder unklare Symbolik, überprüft. Markierte Fehler wurden anschließend korrigiert, um die Einhaltung elektrotechnischer Standards sicherzustellen. Zusätzlich erfolgte eine Layoutanpassung, um die Übersichtlichkeit zu verbessern. Schließlich wurde die zugrundeliegende Norm des ursprünglichen Stromlaufplans identifiziert, recherchiert und in die gewählte Norm DIN EN 60617 "übersetzt." Die überarbeiteten Pläne wurden mit Autodesk Fusion 360 in ein DIN-A3-Format übertragen. Durch die Integration von Titelblöcken und Legenden konnte eine professionelle und lesbare Darstellung sichergestellt werden.

Stromlaufpläne wurden für alle Systemteile erstellt, die zum Zeitpunkt der Veröffentlichung dieser Arbeit finalisiert waren. Diese Pläne entsprechen den Standards der DIN EN 60617, sind einheitlich gestaltet und in einem übergeordneten Gesamtprojekt integriert. Insgesamt wurden fünf Stromlaufpläne erstellt. Aufgrund interner Absprachen wurde die Erstellung der Bestückungspläne auf den nächsten Bearbeitungszeitraum im Sommersemester 2025 verschoben.  

Im Rahmen dieser Arbeiten wurde das Team in die Software Autodesk Fusion 360 eingearbeitet, eine eigene Bibliothek für die Norm DIN EN 60617 erstellt und eine Struktur zur einfachen Erweiterung der Bibliothek implementiert. Zudem wurde sichergestellt, dass neue Teammitglieder unkompliziert zur Projektcloud hinzugefügt werden können. Eine detaillierte Anleitung zur Installation der CAD-Software auf allen relevanten Betriebssystemen wurde ebenfalls bereitgestellt. Der Installationsprozess auf den laborinternen Geräten wurde gestartet.  

Neben den teamspezifischen Aufgaben unterstützte das Team andere Projektgruppen. Dies umfasste beispielsweise das Verladen sowie den Ein- und Ausbau der Batterie in das Fahrzeug und das Bohren zusätzlicher Löcher in die Außenwand des Fahrzeugs, um die Funktionalität der Lüfter zu verbessern.

\section*{Ausblick}
Zwischen dem abgeschlossenen Arbeitszeitraum "Wintersemester 2024" und dem bevorstehenden "Sommersemester 2025" werden die Teams ihre Aufgaben im reduzierten Umfang weiterführen, je nach Bedarf und Möglichkeit. Zudem ist eine TÜV-Abnahme des Fahrzeugs geplant.  

Im kommenden Bearbeitungszeitraum sollen ergänzend zur bestehenden Dokumentation weitere Stromlaufpläne erstellt und integriert werden. Zusätzlich wird die Dokumentation durch Bestückungspläne erweitert. Um die Übersichtlichkeit weiter zu erhöhen, wird die gesamte Dokumentation in einem zentralen Dokument zusammengeführt.  

Weitere Ziele für das nächste Semester umfassen die Modularisierung der Batterie, den Einbau eines Bussystems, die Erweiterung der Temperaturüberwachung, die Modularisierung der Powerbox sowie die Installation eines Cockpit-Displays, das Fahr-, Verbrauchs- und Leistungsdaten anzeigt. Als optionales Ziel wird der Einbau einer Musikanlage in Betracht gezogen, sofern der Fortschritt des Systems dies zulässt.