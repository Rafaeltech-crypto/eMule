\chapter{critical reflection and perspective}
\label{cha:zusammenfassung}

%Auf zwei bis drei Seiten soll auf folgende Punkte eingegangen werden:

%Im Rahmen des Projekts eMule 7.0 wurden verschiedene Ziele für das Wintersemester 2024 definiert, darunter:  
%Im Rahmen des Projekts eMule 7.0 wurden verschiedene Ziele für das Sommersemester 2024 definiert, darunter:
As part of the eMule 7.0 project, a series of objectives were delineated for the summer semester of 2025, including:
\begin{itemize}
%	\item[1.] Stromlaufpläne dokumentieren
%	\item[2.] Bestückungspläne dokumentieren
%	\item[3.] 3D-CAD-Modell des Energiespeichersystems
%	\item[4.] Unterstützung bei der Erstellung, Testing und implementierung des autonomen Fahrens
	\item[1.] Document circuit diagrams
\item[2.] Document assembly plans
\item[3.] 3D CAD model of the energy storage system
\item[4.] Support in the creation, testing and implementation of autonomous driving

\end{itemize}

%Um die Erstellung aktueller Stromlaufpläne effizient zu gestalten, wurden zwei verschiedene Vorgehensweisen angewendet. Bei der Aktualisierung bereits vorhandener Pläne wurde folgender Prozess implementiert:  
%Die bestehenden Pläne wurden zunächst ausgedruckt und systematisch auf Unstimmigkeiten, wie fehlende Verbindungen oder unklare Symbolik, überprüft. Markierte Fehler wurden anschließend korrigiert, um die Einhaltung elektrotechnischer Standards sicherzustellen. Zusätzlich erfolgte eine Layoutanpassung, um die Übersichtlichkeit zu verbessern. Schließlich wurde die zugrundeliegende Norm des ursprünglichen Stromlaufplans identifiziert, recherchiert und in die gewählte Norm DIN EN 60617 "übersetzt." Die überarbeiteten Pläne wurden mit Autodesk Fusion 360 in ein DIN-A3-Format übertragen. Durch die Integration von Titelblöcken und Legenden konnte eine professionelle und lesbare Darstellung sichergestellt werden.
%Das Verfahren der Erstellung von Stromlaufplänen wurde weiter verbessert, indem Skizzen bereits möglichst realitätsnah statt wie bisher nur schematisch erstellt wurden. 
The creation of circuit diagrams has been further refined through the use of sketches that are as realistic as possible, rather than merely schematically as was previously the case.

%Stromlaufpläne wurden für alle Systemteile erstellt, die zum Zeitpunkt der Veröffentlichung dieser Arbeit finalisiert waren. Diese Pläne entsprechen den Standards der DIN EN 60617, sind einheitlich gestaltet und in einem übergeordneten Gesamtprojekt integriert. Insgesamt wurden sechs Stromlaufpläne erstellt. 
The circuit diagrams for all system components were created at the time of publication of this work. These plans comply with the standards of DIN EN 60617, have a standardised design and are integrated into a superordinate overall project. The creation of a total of six circuit diagrams was undertaken.

%Um die Erstellung der Bestückungspläne effizient zu gestalten wurde die folgende Vorgehensweise angewendet:
%Zunächst wurde eine händische Skizze des Systems angefertigt und ausgedruckt. Im ersten Schritt erfolgte eine systematische Überprüfung des vorliegenden Bestückungsplans auf Unstimmigkeiten, wie etwa falsche oder fehlende Verbindungen sowie unklare Symbolik. Im darauffolgenden Schritt erfolgte die Markierung der identifizierten Fehler sowie deren Korrektur. Darüber hinaus wurde eine Layoutanpassung vorgenommen, um die Übersichtlichkeit zu optimieren. Es wurde sichergestellt, dass die Komponenten im Plan an der identischen Position wie im Fahrzeug platziert wurden. Die Darstellung der relevanten Verbindungen der Bauteile erfolgte in der Planung. Im finalen Schritt wurde der überarbeitete Stromlaufplan mit Autodesk Fusion 360 in das DIN-A3-Format übertragen. In diesem Kontext wurden Titelblock und Legende integriert, um die Professionalität und Lesbarkeit zu gewährleisten.
The following procedure was utilised in order to create the assembly plans in an efficient manner:
Initially, a manual sketch of the system was created and subsequently printed. The initial step in the process entailed a systematic examination of the existing assembly plan to identify any potential inconsistencies, including but not limited to incorrect or missing connections and unclear symbols. In the subsequent stage, the identified errors were annotated and rectified. Furthermore, the layout was modified to enhance clarity. It was imperative that the components were placed in the same position in the plan as in the vehicle. The relevant component connections were visualised in the planning stage. In the final step of the process, the revised circuit diagram was transferred to DIN A3 format using Autodesk Fusion 360. In this particular instance, the integration of the title block and legend was a deliberate design choice to ensure the maintenance of a professional and legible aesthetic.

\section{Conclusion on the CAD Design Process}

Using Fusion 360 allowed for a highly structured, iterative, and parametric modeling workflow. All geometric relationships between components were carefully maintained, which significantly simplified later adjustments. The use of parametric sketches and pattern tools proved crucial in efficiently laying out large arrays of repetitive features, such as the cell holders.

The final result is a complete digital twin of the battery system, suited for additive manufacturing and integration in the E-Mule platform. While no physical tests or simulations were performed, the model meets the spatial, mechanical, and packaging criteria outlined at the beginning of the project.

%Die Implementierung des autonomen Fahrens konnte nicht unterstützt werden. Das Team, dessen Hauptaufgabe in der Implementierung bestand, forschte in eine falsche Richtung, sodass es nicht weit genug fortgeschritten war, um die Unterstützung auf dem Niveau unserer Fachkompetenzen in Anspruch nehmen zu können.

The implementation of autonomous driving could not be supported. The team whose main task was the implementation was researching in the wrong direction, so it was not advanced enough to be able to utilise support at the level of our expertise.

%Im Rahmen dieser Arbeiten wurde das Team tiefer in die Software Autodesk Fusion 360 eingearbeitet, die eigene Bibliothek für die Norm DIN EN 60617 erweitert, eine eigene Bibliothek für das CAD-Design erstellt und eine Struktur zur einfachen Erweiterung der Bibliothek implementiert. Darüber hinaus wurde sichergestellt, dass neue Teammitglieder unkompliziert in die Projektcloud integriert werden können.

\section{Hiuer fehlt ne überschrift aber mir ist nix eingefallen}
As part of this project, the team familiarised themselves more deeply with the Autodesk Fusion 360 software. In addition, the in-house library for the DIN EN 60617 standard was expanded, a separate library for CAD design was created, and a structure for easy expansion of the library was implemented. It was also ascertained that new team members can be readily incorporated into the project cloud.
%Eine detaillierte Anleitung zur Installation der CAD-Software auf allen relevanten Betriebssystemen wurde ebenfalls bereitgestellt. Der Installationsprozess auf den laborinternen Geräten wurde gestartet.  

%Neben den teamspezifischen Aufgaben unterstützte das Team andere Projektgruppen. Dies umfasste beispielsweise das Verladen sowie den Ein- und Ausbau der Batterie in das Fahrzeug und !!!Buck!!!

In addition to the team-specific tasks, the team provided support to other project groups. This encompassed, for instance, the processes of loading, installing and removing the battery in the vehicle and \textbf{\underline{!!!Buck!!!}}

\section*{Perspective}
%Zwischen dem abgeschlossenen Arbeitszeitraum "Wintersemester 2024" und dem bevorstehenden "Sommersemester 2025" werden die Teams ihre Aufgaben im reduzierten Umfang weiterführen, je nach Bedarf und Möglichkeit. Zudem ist eine TÜV-Abnahme des Fahrzeugs geplant.  
%
%Im kommenden Bearbeitungszeitraum sollen ergänzend zur bestehenden Dokumentation weitere Stromlaufpläne erstellt und integriert werden. Zusätzlich wird die Dokumentation durch Bestückungspläne erweitert. Um die Übersichtlichkeit weiter zu erhöhen, wird die gesamte Dokumentation in einem zentralen Dokument zusammengeführt.  
%
%Weitere Ziele für das nächste Semester umfassen die Modularisierung der Batterie, den Einbau eines Bussystems, die Erweiterung der Temperaturüberwachung, die Modularisierung der Powerbox sowie die Installation eines Cockpit-Displays, das Fahr-, Verbrauchs- und Leistungsdaten anzeigt. Als optionales Ziel wird der Einbau einer Musikanlage in Betracht gezogen, sofern der Fortschritt des Systems dies zulässt.

%Nach dem abgeschlossenen Sommersemester 2025 gibt es noch zahlreiche Tehmen, denen sich neue Teams in zukünftigen Projekten annehmen können.
%Weiterhin steht eine TÜV Abnahme als übergeordnetes Ziel im Vordergrund.
%Das optionale Ziel des Verbaus einer Musikanlage ist ebenfalls noch present.
%
%Weitere Ziele für die nächsten Bearbeitungszeiträume umfassen:
Following the conclusion of the 2025 summer semester, new teams will have the opportunity to assume responsibility for a number of tasks in future projects.
The overarching objective remains the attainment of TÜV approval.
The option of installing a music system remains.

The following objectives have been delineated for the ensuing processing periods:

\begin{itemize}
%	\item BMS austauschen
%	\item Verkleidung für das Ladeschloss anbringen
%	\item Service durchführen
%	\item Team-E-Mail einrichten
%	\item Cloud implementiern
%	\item Dokumentation stetig aktuell halten
	\item The replacement of the BMS is to be carried out
\item The Affix of a cover for the charging lock
\item The performance of a service is to be undertaken
\item The establishment of an email account for the team
\item The implementation of cloud technology
\item The documentation has to be maintained in a consistently updated state
\item Implementation af a complete eMule 3D Modell
	
\end{itemize}
