\chapter*{Kurzfassung} %*-Variante sorgt dafür, das Abstract nicht im Inhaltsverzeichnis auftaucht

Diese Arbeit begleitet die Erstellung oder Überarbeitung von Stromlaufplänenn eines von konventinellem Verbrennungsmotor auf Elektroantrieb umgebauten Kawaski Mule 610 Fahrzeugs. \newline 
Im Rahmen der Implementierung eines Energiespeichersystems in das umgetaufte \glqq Kawasaki eMule \grqq sollen folgende Ziele erreicht werden: Die Erstellung von aktuellen Stromlauf und Bestückungsplänen der elektrisch/elektronischen Fahrzeugschaltkreise.Eine Vereinheitlichung und Vergemeinschaftung der Pläne des Systems und dessen Einzelkomponenten. Die Erstellung einer Installationsanleitung für Autodesk Fusion 360 auf verschiedenen Betriebssystemen.\newline
Im ersten Schritt wurde die Installationsanleitung erstellt, da die Teammitglieder selbst ebenfalls noch die Software herunterladen müseen, sowie die Installtion auf den Laborendgeräten durchgeführt werden musste. Da die priveten Endgeräte der Teammitglieder auf macOS laufen und die des Labors auf Windows könnten direkt beide Varianten abgedeckt werden. Im nächsten Schritt wurden die Stromlaufpläne in der Softwareumgebung ersellt und vergemeinschaftet.Immer wenn gerade Zeit entbährt werden konnte, wurde den anderen Teams bestmöglich unter die Arme gegriffen.\newline
Insgesamt wurden fünf Stromlaufpläne, entsprechend der DIN EN 60617 realisiert. Im Rahmen der Pläneerstellung wurde eine eigene Bibliothek zur DIN EN 60617 angelegt und eine Struktur zur einfachen Erweiterung der Vergemeinschaftung angelgt. Weiter konnte eine detailierte Anleitung zur Installation der CAD-Softwäre Autodesk Fusion 360 auf den Betriebssystemen macOS und Windows zur Verfügung gestellt werden. Neben den Teamspezifischen Aufgaben konnte die anderen Projektteams bei diversen Aufgaben unterstützt werden.

\clearpage

\chapter*{Abstract} %*-Variante sorgt dafür, das Abstract nicht im Inhaltsverzeichnis auftaucht

English translation of the \glqq Kurzfassung\grqq.

\clearpage