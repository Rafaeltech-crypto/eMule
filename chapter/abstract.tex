
\chapter*{Abstract} %*-Variante sorgt dafür, das Abstract nicht im Inhaltsverzeichnis auftaucht

%This study focused on the development and revision of circuit diagrams for a vehicle that was converted from a conventional internal combustion engine to an electric drive, specifically the Kawasaki Mule 610. As part of the integration of an energy storage system into the modified vehicle model “Kawasaki eMule,” the following objectives were pursued: the creation of updated circuit and component diagrams for the electrical and electronic vehicle systems, the standardization and integration of the system’s plans and its individual components, and the formulation of an installation manual for Autodesk Fusion 360, applicable across different operating systems. \newline The initial phase involved the preparation of the installation manual, as team members needed to download the software and perform installations on the laboratory devices. Given that the team members’ devices operated on macOS and the laboratory devices on Windows, both platforms were addressed within the manual. The subsequent phase involved the creation and unification of circuit diagrams within the software environment. During periods of reduced workload, assistance was provided to other teams as needed. \newline In total, five circuit diagrams were developed in accordance with the DIN EN 60617 standard. As part of the diagram development process, a custom library based on the DIN EN 60617 standard was established, along with a structure designed for efficient expansion and integration of the library. Additionally, a comprehensive installation guide for Autodesk Fusion 360 was compiled for both macOS and Windows operating systems. In addition to fulfilling the team’s specific tasks, support was also extended to other project teams in relation to various tasks.

This thesis dealt with the documentation of the electrical conversion of a vehicle that was converted from a conventional combustion engine to an electric drive, specifically the Kawasaki Mule 610. The following objectives were to be achieved as part of the implementation of an energy storage system in the converted vehicle model ‘Kawasaki eMule’: The creation of current circuit diagrams as well as assembly plans for the electrical/electronic vehicle circuits. Standardisation and integration of the plans for the system and its individual components and the creation of a 3D CAD model (computer-aided design) of the self-developed energy storage system. In addition, the neighbouring team was to be supported in the implementation of autonomous driving. \newline 
The creation of the circuit and assembly plans ran parallel to the creation of the CAD model. Support for autonomous driving was integrated towards the end of the project.
In phases where there was no immediate workload, support was offered by other teams.\newline
A total of six circuit diagrams were created in accordance with the DIN EN (German industrial standard, European standard) 60617 standard and two assembly diagrams. In the course of creating these plans, a separate library was developed on the basis of DIN EN 60617 and a structure was created for the efficient expansion and integration of the library. As part of the CAD model creation, numerous small individual models were created and assembled into a large overall model. A separate library was also implemented for this purpose. In addition to the specific tasks of our own team, we were also able to support other project teams with various tasks.



\chapter*{Kurzfassung} %*-Variante sorgt dafür, das Abstract nicht im Inhaltsverzeichnis auftaucht

%Diese Arbeit befasste sich mit der Erstellung und Überarbeitung von Stromlaufplänen für ein Fahrzeug, das von einem konventionellen Verbrennungsmotor auf einen Elektroantrieb umgerüstet wurde, konkret dem Kawasaki Mule 610. 
Diese Arbeit befasste sich mit der Dokumentation der elektrischen Umsetzung eines Fahrzeugs, das von einem konventionellen Verbrennungsmotor auf einen Elektroantrieb umgerüstet wurde, konkret dem Kawasaki Mule 610.
%Im Rahmen der Implementierung eines Energiespeichersystems in das umgebaute Fahrzeugmodell „Kawasaki eMule“ sollten die folgenden Ziele erreicht werden: die Erstellung von aktuellen Stromlauf- und Bestückungsplänen für die elektrisch/elektronischen Fahrzeugschaltkreise, eine Standardisierung und Integration der Pläne des Systems sowie seiner Einzelkomponenten und die Entwicklung einer Installationsanleitung für Autodesk Fusion 360, die auf verschiedenen Betriebssystemen anwendbar war.
Im Rahmen der Implementierung eines Energiespeichersystems in das umgebaute Fahrzeugmodell „Kawasaki eMule“ sollten die folgenden Ziele erreicht werden: Die Erstellung von aktuellen Stromlaufplänen sowie  Bestückungsplänen für die elektrisch/elektronischen Fahrzeugschaltkreise. Eine Standardisierung und Integration der Pläne des Systems sowie seiner Einzelkomponenten und die Erstellung eines 3D-CAD-Modells (Computer-Aided Design )des selbst entwickelten Energiespeichersystems konstruiert werden. Zusätzlich sollte das Nachbarteam bei der Implementierung des autonomen Fahrens unterstüzt werden. \newline
%Der erste Schritt umfasste die Erstellung der Installationsanleitung. Angesichts der Tatsache, dass die Endgeräte der Teammitglieder mit macOS und die Laborgeräte mit Windows betrieben wurden, wurden beide Varianten in der Anleitung berücksichtigt. Im darauf folgenden Schritt erfolgte die Erstellung und Vereinheitlichung der Stromlaufpläne in der Softwareumgebung. In Phasen, in denen keine unmittelbare Arbeitsbelastung vorlag, wurde die Unterstützung anderer Teams angeboten.
Die Erstellung der Stromlauf- und Bestückungspläne lief parallel zu der Erstellung des CAD-Modells. Die Unterstützung des autonomen Fahrens wurde gegen Ende des Projekt eingebunden.
In Phasen, in denen keine unmittelbare Arbeitsbelastung vorlag, wurde die Unterstützung anderer Teams angeboten.\newline
%Insgesamt wurden fünf Stromlaufpläne gemäß der Norm DIN EN(Deutsche Industrienorm, Europäische Norm) 60617 erstellt. Im Zuge der Erstellung dieser Pläne wurde eine eigene Bibliothek auf Basis der DIN EN 60617 entwickelt und eine Struktur zur effizienten Erweiterung und Integration der Bibliothek geschaffen. Zudem wurde eine detaillierte Anleitung zur Installation der CAD-Software Autodesk Fusion 360 für die Betriebssysteme macOS und Windows ausgearbeitet. Neben den spezifischen Aufgaben des eigenen Teams konnte auch die Unterstützung anderer Projektteams bei verschiedenen Aufgabenstellungen erfolgen.
Insgesamt wurden sechs Stromlaufpläne gemäß der Norm DIN EN(Deutsche Industrienorm, Europäische Norm) 60617 und zwei Bestückungspläne erstellt. Im Zuge der Erstellung dieser Pläne wurde eine eigene Bibliothek auf Basis der DIN EN 60617 entwickelt und eine Struktur zur effizienten Erweiterung und Integration der Bibliothek geschaffen. Im Rahmen der CAD-Modell Erstellung wurden zahlreiche kleine Einzelmodelle erstellt und zu einem großen Gesamtmodell zusammengesetzt. Hierfür wurde ebenfalls eine eigene Bibiliothek implementiert. Neben den spezifischen Aufgaben des eigenen Teams konnte auch die Unterstützung anderer Projektteams bei verschiedenen Aufgabenstellungen erfolgen.



\clearpage