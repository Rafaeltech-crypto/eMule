\chapter{Problemstellung, Ziel und Vorgehensweise der Arbeit}
\label{cha:Problemstellung, Ziel und Vorgehensweise der Arbeit}

Der globale Wandel hin zu nachhaltigeren Mobilitätslösungen ist in vollem Gange. Angesichts steigender Umweltauflagen und wachsendem Bewusstsein für die negativen Auswirkungen fossiler Brennstoffe vollzieht sich ein paradigmatischer Wechsel von konventionellen Verbrennungsmotoren hin zu Elektroantrieben. Dieser technologische Umbruch betrifft nicht nur den privaten Automobilsektor, sondern auch Nutzfahrzeuge und Spezialfahrzeuge, die zunehmend auf elektrische Antriebe umgestellt werden.\autocite{Pischinger2023}
Im Rahmen eines universitären Projekts haben wir ein Kawasaki Mule 610 Fahrzeug erfolgreich von einem Verbrennungsmotor auf einen Elektroantrieb umgerüstet. Dieser Umbau stellte einen ersten Meilenstein dar, der es uns ermöglichte, die Vorteile elektrischer Mobilität in einer praktischen Anwendung zu demonstrieren. Nun soll das Projekt weiterentwickelt werden, um durch gezielte Verbesserungen – wie den Einsatz einer leistungsfähigeren Batterie – die Effizienz und Reichweite des Fahrzeugs zu optimieren und neue Standards in der elektrischen Antriebstechnologie zu setzen.

\section*{Problemstellung}
Das Kawasaki Mule 610 wurde durch unsere Vorgängerjahrgänge von einen auslieferungsgemäß verbauten Verbrennungsmotor auf einen Elektromotorantrieb umgebaut. Das Fahrzug wurde von \glqq Kawasaki Mule 610 \grqq in \glqq Kawasaki eMule \grqq umgetauft. Die Dokumentation des Umbaus wurde in diesem Zuge nur notdürftig bis garnicht und ohne jegliche Vereinheitlichung vorgenommen. Die vorhandene Dokumentation liegt für jeden Stromlaufplan nach einer andren Norm durchgeführt vor. Legenden zu den genutzten Normen sind nicht vorhanden. Die Aktualität der vorliegenden Dokumentation muss durch Abgleiche mit dem Verbaustand des Fahrzeugs und Absprache mit dem Dozenten Herr Khamis Jakob für jeden Stromlaufplan einzeln überprüft werden.2  


 3Im folgenden werden lediglich die Aspekte der Problemstellung für die Aufgabe der Erstellung von Stromlaufplänen und Bestückungsplänen betrachtet. Das erste Problem ergibt sich in der Auswahl eines geeigneten Programmes zum Erstellen der Dokumentation. Die Schwierigkeit besteht hierbei darin, ein Programm zu finden, welches allen gestellten Anforderungen entspricht. Diese sind: 
\begin{itemize}
	\item Das Programm muss möglichst kostengünstig sein, da nur ein begrenztes Budget zur Verfügung steht.
	\item Das Programm muss sowohl für die Betriebssysteme Windows als auch macOS ausgelegt sein, um sicherzustellen, dass jedes Teammitglied optimal arbeiten kann.
	\item Das Programm muss sowohl in der Lage sein Stromlaufpläne als auch Bestückungspläne erstellen zu können, da auf Grund des beschränkten Budgets nicht mehrere Programmlizenzen finanziert werden können.
	\item Das Programm muss die ausgewählte Norm unterstützen, oder die Möglichkeit bieten eigene Bibliotheken mit Bauteilen zu erstellen.
\end{itemize}   
Sollte das Programm die ausgewählte Norm nicht unterstützen und diese muss als eigene Bibliothek angelgt werden, so ist dies mit enormem zeitlichem Mehraufwand verbunden. Dieser Mehraufwand kann eine Gefahr für die angesetzten Zeitziele des Prjektes darstellen. 
Ein weiteres Problem stellt der Umstand, dass noch keins der Teammitglieder sowohls jemals mit CAD-Software, als auch an einem Projekt in diesem Ausmaß  ohne saubere Dokumentation gearbeitet hat.
%Die bereits erwähnte Vereinheitlichung nach DIN Norm bietet noch weitere Probleme. Im ersten Schritt muss eine allgemeine Dokumentation der Norm vorliegen anhand welcher die Dokumentation des Systems erstellt werden kann. Im zweiten Schritt müssen Dokumentationen der für die bisherige Systemdokumentation verwendeten Normen vorliegen, um erkennen zu könne um welche Teile es sich handelt. Ein weiteres Problem stellt die Erstellung einer eigenen Bibliothek für die gewählte Norm dar, da dies eine sehr umfangreiche Aufgabe ist für welche sich intensiv in das Program eingearbeitet werden muss.

\section*{Zielsetzung}
Für das Team, welches für die \glqq Erstellung der Stromlauf- und Bestückugspläne des Fahrzeugs \grqq zuständig ist werden folgende Ziele für den ersten Arbeitszeitraum im Wintersemester 2024 definiert.
\begin{itemize}
	\item vergemeinschaftung der vorhandenen Pläne nach DIn norm
	\item erstellen neuer pläne(Schalt, bestückungs, stromlauf, usw.) nach DIN Norm
	Erstellung von aktuellen Stromlauf und Bestückungsplänen der elektrisch/elektronischen Fahrzeugschaltkreise
	\item[7.] Vereinheitlichung und Vergemeinschaftung der Pläne des Systems und dessen Einzelkomponenten
	\item[8.] Erstellung einer Installationsanleitung für Autodesk Fusion 360 auf sämtlichen Betriebssystemen
\end{itemize}


\section*{Geplante Vorgehensweise}
	 Wie sind wir vorgegangen
\begin{itemize}
	\item geeignetes Program gesucht
	\item eingearbeitet
	\item eigene Bibs erstellt
	\item alte pläne geordnet und brauchbare in din norm übersetzt
	\item neue Pläne gezeichnet
\end{itemize}
Das Team muss in verschiedene Gruppen aufgeteilt werden, um die verschiedenen Teilaufgaben möglichst effizient bearbeiten zu können.
2Im ersten Schritt muss sich das gesamte eMule-Team einen Überblick über das Fahrzeug und dessen Zustand verschafften. Dieser Überblick umfasst sowohl den mechanischen sowie elektrisch/elektronischen Aufbauzustand des Gesamtfahrzeugs und der einzelnen Komponenten. Im nächsten Schritt wählt jedes Teilteam sein eigenes weiteres Vorgehen. In dieser Arbeit wird nur das Vorgehen des Teams zur \glqq Erstellung der Stromlauf- und Bestückugspläne des Fahrzeugs \grqq betrachtet. 


Muss die vorhandene Dokumentation eingesehen werden. Hier wird überprüft, welche Teile der Dokumentation dem aktuellen Aufbauzustand entsprechen. Die Teile der Dokumentation, welche nicht dem aktuellen Stand entsprechen, müssen verworfen werden. Auf Basis der vorhandenen Dokumentation und des Aufbauzustandes des Fahrzeuges muss eine neue gesamtheitliche Dokumentation des Gesamtsystems erstellt werden. Um die Dokumentation einheitlich zu gestalten muss sich auf eine Norm festgelegt werden, nach dieser die neue Dokumentation erstellt wird.3 








	
