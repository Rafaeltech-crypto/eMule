\chapter{Intro}
\label{cha:Problemstellung, Ziel und Vorgehensweise der Arbeit}

%Der globale Wandel hin zu nachhaltigeren Mobilitätslösungen ist in vollem Gange. Angesichts steigender Umweltauflagen und wachsendem Bewusstsein für die negativen Auswirkungen fossiler Brennstoffe vollzieht sich ein paradigmatischer Wechsel von konventionellen Verbrennungsmotoren hin zu Elektroantrieben. Dieser technologische Umbruch betrifft nicht nur den privaten Automobilsektor, sondern auch Nutzfahrzeuge und Spezialfahrzeuge, die zunehmend auf elektrische Antriebe umgestellt werden.
The global shift towards more sustainable mobility solutions is well underway. In light of mounting environmental regulations and growing awareness of the detrimental impact of fossil fuels, we are witnessing a paradigm shift away from traditional combustion engines towards electric drives. This technological upheaval is affecting not only the private automotive sector, but also commercial and specialised vehicles, which are increasingly being converted to electric power.\autocite{Pischinger2023}
%Im Rahmen eines universitären Projekts haben wir ein Kawasaki Mule 610 Fahrzeug erfolgreich von einem Verbrennungsmotor auf einen Elektroantrieb umgerüstet. Dieser Umbau stellte einen ersten Meilenstein dar, der es uns ermöglichte, die Vorteile elektrischer Mobilität in einer praktischen Anwendung zu demonstrieren. Nun soll das Projekt weiterentwickelt werden, um durch gezielte Verbesserungen – wie den Einsatz einer leistungsfähigeren Batterie – die Effizienz und Reichweite des Fahrzeugs zu optimieren und neue Standards in der elektrischen Antriebstechnologie zu setzen.
As part of a university project, we successfully converted a Kawasaki Mule 610 vehicle from a combustion engine to an electric drive. This conversion was a first milestone that enabled us to demonstrate the advantages of electric mobility in a practical application. The project is now to be developed further in order to optimise the efficiency and range of the vehicle through targeted improvements - such as the use of a more powerful battery - and to set new standards in electric drive technology.
\clearpage
\section*{Problem statement}
%Das Kawasaki Mule 610 wurde durch unsere Vorgängerjahrgänge von einen auslieferungsgemäß verbauten Verbrennungsmotor auf einen Elektromotorantrieb umgebaut. Das Fahrzeug wurde von \glqq Kawasaki Mule 610\grqq{} in \glqq Kawasaki eMule\grqq{} umgetauft. Die Dokumentation des Umbaus wurde in diesem Zuge nur notdürftig bis garnicht und ohne jegliche Vereinheitlichung vorgenommen. Die vorhandene Dokumentation liegt für jeden Stromlaufplan nach einer anderen Norm durchgeführt vor. Legenden zu den genutzten Normen sind nicht vorhanden. Die Aktualität der vorliegenden Dokumentation muss durch Abgleiche mit dem Verbaustand des Fahrzeugs und Absprache mit dem Dozenten Herr Khamis Jakob für jeden Stromlaufplan einzeln überprüft werden.  
Our predecessors converted the \glqq Kawasaki Mule 610\grqq{} from an internal combustion engine to an electric motor drive. The vehicle was renamed the \glqq Kawasaki eMule\grqq{}. The conversion documentation was poorly executed, if at all, and was not standardised. The existing documentation for each circuit diagram is according to a different standard. There are no legends for the standards used. The accuracy of the existing documentation must be verified for each circuit diagram by comparing it with the vehicle's installation status and consulting the instructor, Mr Khamis Jakob.


 %Im folgenden werden lediglich die Aspekte der Problemstellung für die Aufgabe der Erstellung von Stromlaufplänen und Bestückungsplänen betrachtet. Das erste Problem ergibt sich in der Auswahl eines geeigneten Programmes zum Erstellen der Dokumentation. Die Schwierigkeit besteht hierbei darin, ein Programm zu finden, welches allen gestellten Anforderungen entspricht. Diese sind: 
 The following considers only the aspects of the problem relating to the creation of circuit diagrams, assembly plans and CAD-models. The first issue is selecting a suitable software for creating the documentation. The difficulty here is finding a programme that meets all the requirements. These are:
\begin{itemize}
	\item %Das Programm muss möglichst kostengünstig sein, da nur ein begrenztes Budget zur Verfügung steht. 
	The software must be as cost-effective as possible, as only a limited budget is available.
	\item %Das Programm muss sowohl für die Betriebssysteme Windows als auch \newline macOS ausgelegt sein, um sicherzustellen, dass jedes Teammitglied optimal arbeiten kann.
	The software must be designed for both Windows and macOS operating systems to ensure that every team member is able to work optimally.
	\item %Das Programm muss sowohl in der Lage sein Stromlaufpläne als auch Bestückungspläne erstellen zu können, da auf Grund des beschränkten Budgets nicht mehrere Programmlizenzen finanziert werden können.
	The software must be able to create circuit diagrams, assembly plans as well as CAD-models, as several software licences cannot be financed due to the limited budget.
	\item %Das Programm muss die ausgewählte Norm unterstützen, oder die Möglichkeit bieten eigene Bibliotheken mit Bauteilen zu erstellen.
	The software must support the selected standard or offer the option of creating your own libraries with components.
\end{itemize}   
%Sollte das Programm die ausgewählte Norm nicht unterstützen und diese muss als eigene Bibliothek angelgt werden, so ist dies mit enormem zeitlichem Mehraufwand verbunden. Dieser Mehraufwand kann eine Gefahr für die angesetzten Zeitziele des Projektes darstellen. 
%Ein weiteres Problem stellt der Umstand, dass noch keins der Teammitglieder sowohls jemals mit Computer-Aided Design (CAD) Software, als auch an einem Projekt in diesem Ausmaß  ohne saubere Dokumentation gearbeitet hat.
If the software does not support the selected standard, creating a separate library would involve an enormous amount of additional time and effort. This additional work could jeopardise the project's time targets.
Another issue is that none of the team members have ever worked with CAD software or on a project of this scale without proper documentation.
%Die bereits erwähnte Vereinheitlichung nach DIN Norm bietet noch weitere Probleme. Im ersten Schritt muss eine allgemeine Dokumentation der Norm vorliegen anhand welcher die Dokumentation des Systems erstellt werden kann. Im zweiten Schritt müssen Dokumentationen der für die bisherige Systemdokumentation verwendeten Normen vorliegen, um erkennen zu könne um welche Teile es sich handelt. Ein weiteres Problem stellt die Erstellung einer eigenen Bibliothek für die gewählte Norm dar, da dies eine sehr umfangreiche Aufgabe ist für welche sich intensiv in das Program eingearbeitet werden muss.

\section*{Objective}
%Für das Team TFE  werden folgende Ziele für den zweiten Arbeitszeitraum im Sommersemester 2025 definiert:
The following objectives are defined for the \glqq TFE team\grqq{} for the second working period in the summer semester of 2025:
\begin{itemize}
	\item %Erstellung von aktuellen Stromlauf und Bestückungsplänen der \newline elektrisch/elektronischen Fahrzeugschaltkreise
	Creation of current circuit diagrams and assembly plans for the \newline  electrical/electronic vehicle circuits
	\item %Vereinheitlichung und Vergemeinschaftung der Pläne des Systems und dessen Einzelkomponenten
	Standardisation and communitisation of the system's plans and its individual components
	%\item Erstellung einer Installationsanleitung für die ausgewählte CAD-Software für verschiedene Betriebssysteme
	\item Construction of a 3D CAD model for the energy storage system
	\item Supporting the implementation of autonomous driving
	\item 
	\textbf{PART BUCKY CAD}
\end{itemize}
\clearpage

\section*{Planned procedure}
%Um die Ziele bestmöglich umsetzen zu können muss das eMule-Team in verschiedene Kleingruppen aufgeteilt werden. So können die verschiedenen Teilaufgaben möglichst effizient bearbeiten werden. Im ersten Schritt muss sich das gesamte eMule-Team einen Überblick über das Fahrzeug und dessen Zustand verschafften. Dieser Überblick umfasst sowohl den mechanischen sowie elektrisch/elektronischen Aufbauzustand des Gesamtfahrzeugs und der einzelnen Komponenten. Im nächsten Schritt wählt jedes Teilteam sein eigenes weiteres Vorgehen. In dieser Arbeit wird nur das Vorgehen des Teams zur \glqq Erstellung der Stromlauf- und Bestückugspläne des Fahrzeugs\grqq{} näher betrachtet. \newline
%Im ersten Schritt wird eine passende CAD-Software ausgewählt. Es wird eine Online-Recherche zu möglichen Optionen durchgeführt. Diese werden dann auf die folgenden Kriterien geprüft:

%Um die Ziele bestmöglich umsetzen zu können muss das eMule-Team in verschiedene Kleingruppen aufgeteilt werden. So können die verschiedenen Teilaufgaben möglichst effizient bearbeiten werden. Im ersten Schritt muss sich das gesamte eMule-Team einen Überblick über das Fahrzeug und dessen Zustand verschafften. Dieser Überblick umfasst sowohl den mechanischen sowie elektrisch/elektronischen Aufbauzustand des Gesamtfahrzeugs und der einzelnen Komponenten. Hierbei werden sowohl der aktuelle Stand, als auch zukünftige Potenziale aufgezeigt. Im nächsten Schritt werden die Aufgaben für jedes Team neu definiert bzw. fortgeführt. In dieser Arbeit werden nur die Aufgaben des \glqq TFE team\grqq{} betrachtet. \newline

To achieve the goals in the best possible way, the eMule team should be divided into smaller groups. This enables the various subtasks to be completed as efficiently as possible. Firstly, the entire eMule team must gain an overview of the vehicle and its condition. This includes the mechanical and electrical/electronic condition of the vehicle as a whole and its individual components. Both the current status and future potential are identified. Next, the tasks for each team are redefined or continued. In this work, only the tasks of the \glqq TFE team\grqq{} are considered. \newline
%\begin{itemize}
%	\item Preis,
%	\item Systemkompatibilität,
%	\item Funktionsangebot,
%	\item Implementierungsmöglichkeiten.
%\end{itemize} Nach der Prüfung sollten zwei bis drei Programme in der näheren Auswahl stehen, welche dann nochmals gegeneinander verglichen werden, um die bestmögliche Option auszuwählen.\newline
%Ist die Auswahl der Software getroffen, soll die Erstellung der Installationsanleitung abgearbeitet werden. Zeitgleich soll die Software sowohl auf den Laborendgeräten als auch auf den für die Nutzung geplanten privaten Endgeräten installiert werden. \newline
%Im ersten Schritt soll eine Einarbeitung in neue Funktionen der Software in Eigenverantwortung durchgeführt werden. Dieser Arbeitsschritt soll sicherstellen, dass bei der späteren Arbeit mit dem Programm möglichst wenig Zeit verloren wird. Im Zuge der Einarbeitung soll auch die Projektumgebung zur späteren Vergemeinschaftung angelegt werden. Diesen dient ebenfalls als Digitale Datenbank des Projekts. \newline

The initial step is to acquaint oneself with the novel functionalities of the software, undertaking this endeavour with full responsibility. This step is intended to ensure that as little time as possible is lost when working with the programme at a later stage. In the course of the familiarisation process, it is also recommended that the project environment be created for the purpose of subsequent sharing. The project's digital database is also served by this. \newline
In the process of devising the plans, it is imperative to exercise due diligence to ensure that the numbering of the components is sequential and that the newly formulated plans are incorporated into the extant documentation. The predefined standard is utilised to ensure the consistency of the plans.\newline
The creation of assembly plans involves the examination and comparison of existing documentation with the current assembly status. Documentation that does not correspond to this standard must be disregarded. It is imperative that a new, comprehensive description of the overall system is created. This description must be based on the existing documents and the vehicle's assembly status.\newline
\textbf{PART BUCKY CAD}
\newline
In order to standardise the individual plans, they are subjected to further review to ensure clarity. If necessary, the plans are adapted and transferred to a DIN A3 format, complete with title block and legend.
%Bei der Erstellung der Pläne muss darauf geachtet werden, dass die Numerierungen der Bauteile fortlaufend sind und die neuen Pläne sich in die vorhandene Dokumentation eingliedern. Die bereits festgelegte Norm wird weiterhin verwendet um die Einheit der Pläne sicherzustellen.\newline
%Um die Bestückungspläne zu erstellen werden vorhandene Unterlagen eingesehen und mit dem aktuellen Aufbauzustand verglichen. Dokumentationen, die diesem nicht entsprechen müssen verworfen werden. Auf Basis der vorhandenen Unterlagen und des Aufbauzustandes des Fahrzeugs muss eine neue, umfassende Beschreibung des Gesamtsystems erstellt werden.\newline
%\textbf{PART BUCKY CAD}
%Zur Vergemeinschaftlichung der einzelnen Pläne werden diese nochmals auf Übersichtlichkeit überprüft und gegebenenfalls angepasst sowie in ein DIN A3 Format mit Titelblock und Legende übertragen.

%Der erste Schritt zur eigentlichen Erstellung der Pläne besteht darin, die vorhandenen Unterlagen einzusehen. Hier wird überprüft, welche Teile der Aufzeichnungen dem aktuellen Aufbauzustand entsprechen. Die Teile der Aufzeichnungen, welche nicht dem aktuellen Stand entsprechen, müssen verworfen werden. Auf Basis der vorhandenen Unterlagen und des Aufbauzustandes des Fahrzeugs muss eine neue, umfassende Beschreibung des Gesamtsystems erstellt werden. Um die Unterlagen einheitlich zu gestalten, muss eine Norm festgelegt werden, nach der die neue Beschreibung erstellt wird.\newline
%Zur Vergemeinschaftlichung der einzelnen Pläne werden diese nochmals auf Übersichtlichkeit überprüft und gegebenenfalls angepasst sowie in ein DIN A3 Format mit Titelblock und Legende übertragen.








	
